% This file was converted from HTML to LaTeX with Nathan Torkington's
% html2latex program
% Version 0.9c
\documentstyle[12pt,rfc]{article}
\i-d
\title{HTML 2.0XXX}
\author{dwc}
\address{mit/w3c}
\pubdate{june 1995}
\begin{document}

\section*{Introduction}\par 
The HyperText Markup Language (HTML) is a simple data format
used to create hypertext documents that are portable from one platform
to another.  HTML documents are SGML documents with generic semantics
that are appropriate for representing information from a wide range of
domains.
\par \subsection*{Scope}\par 
HTML has been in use by the World-Wide Web (WWW) global information
initiative since 1990.  This specification corresponds to the
capabilities of HTML in common use prior to June 1994 and referred to
as "HTML 2.0".
\par \par 
HTML is an application of ISO Standard 8879:1986 {\it Information
Processing Text and Office Systems;  Standard Generalized Markup
Language} (SGML).  The HTML Document Type Definition (DTD) is a
formal definition of the HTML syntax in terms of SGML.
\par \par 
This specification also defines HTML as an Internet Media Type[IMEDIA]
and MIME Content Type[MIME] called {\tt `text/html'}.  As such, it
defines the semantics of the HTML syntax and how that syntax should be
interpreted by user agents.
\par \subsection*{Conformance}\par 
This specification governs the syntax of HTML documents and the
behaviour of HTML user agents.
\par \subsubsection*{Documents}\par 
A document is a conforming HTML document only if:
\par \begin{itemize}\item 
It is a conforming SGML document, and it conforms to the HTML DTD
(see section HTML DTD).
(1)\item 
It conforms to the application conventions in this
specification.  For example, the value of the {\it HREF} attribute of
the {\it A} element must conform to the URI syntax.
\item 
Its document character set includes ANSI/ISO 8859-1 and agrees with
ISO/IEC 10646-1;  that is, each code position listed in section The ANSI/ISO 8859-1 Coded Character Set is included, and each code position in the document
character set is mapped to the same character as ISO10646 designates
for that code position.
(2)\end{itemize}\subsubsection*{Feature Test Entities}\par 
The HTML DTD defines a standard HTML document type and several
variations, based on feature test entities:
\par \begin{description}\item[{\it HTML.Recommended}]
Certain features of the language are
necessary for compatibility with widespread usage, but they may
compromise the structural integrity of a document.  This feature test
entity enables a more prescriptive document type definition that
eliminates those features.
For example, in order to preserve the structure of a document, an
editing user agent may translate HTML documents to the recommended
subset, or it may require that the documents be in the recommended
subset for import.
\item[{\it HTML.Deprecated}]
Certain features of the language are
necessary for compatibility with earlier versions of the
specification, but they tend to be used and implemented inconsistently,
and their use is deprecated.  This feature test entity enables a
document type definition that eliminates these features.
Documents generated by tranlation software or editing software should not
contain these idioms.
\end{description}\subsubsection*{User Agents}\par 
An HTML user agent conforms to this specification if:
\par \begin{itemize}\item 
It parses the characters of an HTML document into data characters
and markup according to [SGML].
(3)\item 
It supports the {\tt `ISO-8859-1'} character encoding scheme and
processes each character in the ISO Latin Alphabet No.  1 as specified
in section The ISO Latin 1 Character Repertoire.
(4)\item 
It behaves identically for documents whose parsed token sequences
are identical.
For example, comments and the whitespace in tags disappear during
tokenization, and hence they do not influence the behaviour of
conforming user agents.
\item 
It allows the user to traverse (or at least attempt to traverse,
resources permitting) all hyperlinks from {\it A} elements in an HTML
document.
\item 
It allows the user to express all form field values specified in
an HTML document and to (attempt to) submit the values as requests to
information services.
\end{itemize}\section*{HTML as an Application of SGML}\par 
HTML is an application of ISO 8879:1986 -- Standard
Generalized Markup Language (SGML).  SGML is a system for defining
structured document types and markup languages to represent instances
of those document types[SGML].  The public text -- DTD and SGML
declaration -- of the HTML document type definition are provided in
section HTML Public Text.
\par \par 
The term {\it HTML} refers to both the document type defined here and
the markup language for representing instances of this document type.
\par \subsection*{SGML Documents}\par 
An HTML document is an SGML document;  that is, a sequence of
characters organized physically into a set of entities, and logically
as a hierarchy of elements.
\par \par 
The first production of the SGML grammar separates an SGML document
into three parts:  an SGML declaration, a prologue, and an
instance.  For the purposes of this specification, the prologue is a
DTD.  This DTD describes another grammar:  the start symbol is given in
the doctype declaration, the terminals are data characters and tags,
and the productions are determined by the element declarations.  The
instance must conform to the DTD, that is, it must be in the language
defined by this grammar.
\par \par 
The SGML declaration determines the lexicon of the grammar.  It
specifies the document character set, which determines a
character repertoire that contains all characters that occur in all
text entities in the document, and the code positions associated with
those characters.
\par \par 
The SGML declaration also specifies the syntax-reference character set
of the document, and a few other parameters that bind the abstract
syntax of SGML to a concrete syntax.  This concrete syntax determines
how the sequence of characters of the document is mapped to a sequence
of terminals in the grammar of the prologue.
\par \par 
For example, consider the following document:
\par $<$!DOCTYPE html PUBLIC "-//IETF//DTD HTML 2.0//EN"$>$
$<$title$>$Parsing Example$<$/title$>$
$<$p$>$Some text. $<$em$>$\&\#42;wow\&\#42;$<$/em$>$$<$/p$>$
\par 
An HTML user agent should use the SGML declaration that is given in
section SGML Declaration for HTML.  According to its document character set,
{\tt `\&\#42;'} refers to an asterisk character.
\par \par 
The instance above is regarded as the following sequence of terminals:
\par \begin{enumerate}\item 
TITLE start-tag
\item 
data characters:  "Parsing Example"
\item 
TITLE end-tag
\item 
P start-tag
\item 
data characters "Some text.  "
\item 
EM start-tag
\item 
"*wow*"
\item 
EM end-tag
\item 
P end-tag
\end{enumerate}\par 
The start symbol of the DTD grammar is HTML, and the productions
are given in the public text identified by {\tt `-//IETF//DTD HTML 2.0//EN'} (section HTML DTD).  Hence the terminals above parse as:
\par    HTML
    |
    $\backslash$-HEAD
    |  |
    |  $\backslash$-TITLE
    |      |
    |      $\backslash$-$<$TITLE$>$
    |      |
    |      $\backslash$-"Parsing Example"
    |      |
    |      $\backslash$-$<$/TITLE$>$
    |
    $\backslash$-BODY
      |
      $\backslash$-P
        |
        $\backslash$-$<$P$>$
        |
        $\backslash$-"Some text. "
        |
        $\backslash$-EM
        |  |
        |  $\backslash$-$<$EM$>$
        |  |
        |  $\backslash$-"*wow*"
        |  |
        |  $\backslash$-$<$/EM$>$
        | 
        $\backslash$-$<$/P$>$
\subsection*{HTML Lexical Syntax}\par 
SGML specifies an abstract syntax and a reference concrete
syntax.  Aside from certain quantities and capacities (e.g.  the limit
on the length of a name), all HTML documents use the reference
concrete syntax.  In particular, all markup characters are in the
repertoire of ISO 646 IRV.  Data characters are drawn from the document
character set (see section Characters, Words, and Paragraphs).
\par \par 
A complete discussion of SGML parsing, e.g.  the mapping of a sequence
of characters to a sequence of tags and data, is left to the SGML
standard[SGML].  This section is only a summary.
\par \subsubsection*{Data Characters}\par 
Any sequence of characters that do not constitute markup (see
9.6 "Delimiter Recognition" of [SGML]) are mapped directly to
strings of data characters.  Some markup also maps to data character
strings.  Numeric character references also map to single-character
strings, via the document character set.  Each reference to one of the
general entities defined in the HTML DTD also maps to a
single-character string.
\par \par 
For example,
\par abc\&lt;def    =$>$ "abc","$<$","def"
abc\&\#60;def   =$>$ "abc","$<$","def"
\par 
Note that the terminating semicolon is only necessary when the
character following the reference would otherwise be recognized as
markup:
\par abc \&lt def     =$>$ "abc ","$<$"," def"
abc \&\#60 def    =$>$ "abc ","$<$"," def"
\par 
And note that an ampersand is only recognized as markup when it
is followed by a letter or a {\tt `\#'} and a digit:
\par abc \& lt def    =$>$ "abc \& lt def"
abc \&\# 60 def    =$>$ "abc \&\# 60 def"
\par 
A useful technique for translating plain text to HTML is to replace
each '$<$', '\&', and '$>$' by an entity reference or numeric character
reference as follows:
\par                  ENTITY      NUMERIC
       CHARACTER REFERENCE   CHAR REF     CHARACTER DESCRIPTION
         \&       \&amp;       \&\#38;        Ampersand 
         $<$       \&lt;        \&\#60;        Less than
         $>$       \&gt;        \&\#62;        Greater than
\par (5)\par \subsubsection*{Tags}\par 
Tags delimit elements such as headings, paragraphs, lists, character
highlighting, and links.  Most HTML elements are identified in a
document as a start-tag, which gives the element name and attributes,
followed by the content, followed by the end tag.  Start-tags are
delimited by {\tt `$<$'} and {\tt `$>$'};  end tags are delimited by
{\tt `$<$/'} and {\tt `$>$'}.  An example is:
\par $<$H1$>$This is a Heading$<$/H1$>$
\par 
Some elements only have a start-tag without an end-tag.  For example,
to create a line break, you use the {\tt `$<$BR$>$'} tag.  Additionally,
the end tags of some other elements, such as Paragraph
({\tt `$<$/P$>$'}), List Item ({\tt `$<$/LI$>$'}), Definition Term
({\tt `$<$/DT$>$'}), and Definition Description ({\tt `$<$DD$>$'})
elements, may be omitted.
\par \par 
The content of an element is a sequence of data character strings and nested 
elements.  Some elements, such as anchors, cannot be nested.  Anchors 
and character highlighting may be put inside other constructs.  See
the HTML DTD, section HTML DTD for full details.
(6)\par \subsubsection*{Names}\par 
A name consists of a letter followed by up to 71 letters, digits,
periods, or hyphens.  Element names are not case sensitive, but entity
names are.  For example, {\tt `$<$BLOCKQUOTE$>$'},
{\tt `$<$BlockQuote$>$'}, and {\tt `$<$blockquote$>$'} are equivalent,
whereas {\tt `\&amp;'} is different from {\tt `\&AMP;'}.
\par \par 
In a start-tag, the element name must immediately follow the tag 
open delimiter {\tt `$<$'}.
\par \subsubsection*{Attributes}\par 
In a start-tag, white space and attributes are allowed between the
element name and the closing delimiter.  An attribute typically
consists of an attribute name, an equal sign, and a value, though
some attributes may be just a value.  White space is allowed around
the equal sign.
\par \par 
The value of the attribute may be either:
\par \begin{itemize}\item 
A string literal, delimited by single quotes or double quotes and
not containing any occurrences of the delimiting character.
(7)\item 
A name token (a sequence of letters, digits, periods, or hyphens).
(8)\end{itemize}\par 
In this example, {\it img} is the element name, {\it src} is the
attribute name, and {\tt `http://host/dir/file.gif'} is the
attribute value:
\par $<$img src='http://host/dir/file.gif'$>$
\par 
A useful technique for computing an attribute value literal for a
given string is to replace each quote and space character by an
entity reference or numeric character reference as follows:
\par                  ENTITY      NUMERIC
       CHARACTER REFERENCE   CHAR REF     CHARACTER DESCRIPTION
         TAB                 \&\#9;         Tab
         LF                  \&\#10;        Line Feed
         CR                  \&\#13;        Carriage Return
                             \&\#32;        Space
         "       \&quot;      \&\#34;        Quotation mark 
         \&       \&amp;       \&\#38;        Ampersand 
\par 
For example:
\par $<$IMG SRC="image.jpg" alt="First \&quot;real\&quot; example"$>$
\par 
Note that the SGML declaration in section 13.3 limits the length of
an attribute value to 1024 characters.
\par \par 
Attributes such as ISMAP and COMPACT may be written using a minimized
syntax.  The markup:
\par $<$UL COMPACT="compact"$>$
\par 
can be written using a minimized syntax:
\par $<$UL COMPACT$>$
\par (9)\par \subsubsection*{Comments}\par 
To include comments in an HTML document, use a comment declaration.  A
comment declaration consists of {\tt `$<$!'} followed by zero or more
comments followed by {\tt `$>$'}.  Each comment starts with {\tt `--'} and
includes all text up to and including the next occurrence of
{\tt `--'}.  In a comment declaration, white space is allowed after
each comment, but not before the first comment.  The entire comment
declaration is ignored.
(10)\par \par 
For example:
\par $<$!DOCTYPE HTML PUBLIC "-//IETF//DTD HTML 2.0//EN"$>$
$<$HEAD$>$
$<$TITLE$>$HTML Comment Example$<$/TITLE$>$
$<$!-- Id: html-sgml.sgm,v 1.5 1995/05/26 21:29:50 connolly Exp  --$>$
$<$!-- another -- -- comment --$>$
$<$!$>$
$<$/HEAD$>$
$<$BODY$>$
$<$p$>$ $<$!- not a comment, just regular old data characters -$>$
\subsection*{HTML Public Text Identifiers}\par 
To identify information as an HTML document conforming to this
specification, each document should start with one of the following
document type declarations.
\par $<$!DOCTYPE HTML PUBLIC "-//IETF//DTD HTML 2.0//EN"$>$
\par 
This document type declaration refers to the HTML DTD in section HTML DTD.
(11)\par $<$!DOCTYPE HTML PUBLIC "-//IETF//DTD HTML 2.0 Level 2//EN"$>$
\par 
This document type declaration also refers to the HTML DTD which
appears in section HTML DTD.
\par $<$!DOCTYPE HTML PUBLIC "-//IETF//DTD HTML 2.0 Level 1//EN"$>$
\par 
This document type declaration refers to the level 1 HTML DTD in
section Strict HTML DTD.  Form elements must not occur in level 1
documents.
\par $<$!DOCTYPE HTML PUBLIC "-//IETF//DTD HTML 2.0 Strict//EN"$>$
$<$!DOCTYPE HTML PUBLIC "-//IETF//DTD HTML 2.0 Strict Level 1//EN"$>$
\par 
These two document type declarations refer to the HTML DTD in section Strict HTML DTD and section Strict Level 1 HTML DTD.  They refer to the more
structurally rigid definition of HTML.
\par \par 
HTML user agents not required to support other document types, but
they may.  In particular, they may support other formal public
identifiers, or other document types altogether.  They may support
an internal declaration subset with supplemental entity,
element, and other markup declarations, or they may not.
\par \subsection*{Example HTML Document}$<$!DOCTYPE HTML PUBLIC "-//IETF//DTD HTML 2.0//EN"$>$
$<$HTML$>$
$<$!-- Here's a good place to put a comment. --$>$
$<$HEAD$>$
$<$TITLE$>$Structural Example$<$/TITLE$>$
$<$/HEAD$>$$<$BODY$>$
$<$H1$>$First Header$<$/H1$>$
$<$P$>$This is a paragraph in the example HTML file. Keep in mind 
that the title does not appear in the document text, but that 
the header (defined by H1) does.$<$/P$>$
$<$OL$>$
$<$LI$>$First item in an ordered list.
$<$LI$>$Second item in an ordered list.
  $<$UL COMPACT$>$
  $<$LI$>$ Note that lists can be nested;
  $<$LI$>$ Whitespace may be used to assist in reading the 
       HTML source.
  $<$/UL$>$
$<$LI$>$Third item in an ordered list.
$<$/OL$>$
$<$P$>$This is an additional paragraph. Technically, end tags are 
not required for paragraphs, although they are allowed. You can 
include character highlighting in a paragraph. $<$EM$>$This sentence 
of the paragraph is emphasized.$<$/EM$>$ Note that the \&lt;/P\&gt; 
end tag has been omitted.
$<$P$>$
$<$IMG SRC ="triangle.xbm" alt="Warning: "$>$
Be sure to read these $<$b$>$bold instructions$<$/b$>$.
$<$/BODY$>$$<$/HTML$>$
\section*{HTML as an Internet Media Type}\par 
An HTML user agent allows users to interact with resources which have
HTML representations.  At a minimum, it must allow users to examine and
navigate the content of HTML level 1 documents.  HTML user agents
should be able to preserve all formatting distinctions represented in
an HTML document, and be able to simultaneously present resources
referred to by IMG elements (they may ignore some formatting
distinctions or IMG resources at the request of the user).  Conforming
HTML user agents should support form entry and submission.
\par \subsection*{text/html media type}\par 
This specification defines the Internet Media Type[IMEDIA] (formerly
referred to as the Content Type[MIME]) called {\tt `text/html'}.  The
following is to be registered with [IANA].
\par \begin{description}\item[{\it Media Type name}]
text
\item[{\it Media subtype name}]
html
\item[{\it Required parameters}]
none
\item[{\it Optional parameters}]
level, charset
\item[{\it Encoding considerations}]
any encoding is allowed
\item[{\it Security considerations}]
see section Security Considerations\end{description}\par 
The optional parameters are defined as follows:
\par \begin{description}\item[{\it Level}]
The level parameter specifies the feature set used in the
document.  The level is an integer number, implying that any features
of same or lower level may be present in the document.  Level 1 is all
features defined in this specification except those that require the
{\it FORM} element.  Level 2 includes form processing.  Level 2 is the
default.
\item[{\it Charset}]
The charset parameter (as defined in section 7.1.1 of
RFC 1521[MIME]) may be given to specify the character encoding scheme
used to represent the HTML document as a sequence of octets.  The
default value is outside the scope of this specification;  but for
example, the default is {\tt `US-ASCII'} in the context of MIME mail,
and {\tt `ISO-8859-1'} in the context of HTTP.
\end{description}\subsection*{HTML Document Representation}\par 
A message entity with a content type of {\tt `text/html'} represents
an HTML document, consisting of a single text entity.  The
{\tt `charset'} parameter (whether implicit or explicit) identifies a
character encoding scheme.  The text entity consists of the characters
determined by this character encoding scheme and the octets of the body of
the message entity.
\par \subsubsection*{Undeclared Markup Error Handling}\par 
To facilitate experimentation and interoperability between
implementations of various versions of HTML, the installed base of
HTML user agents supports a superset of the HTML 2.0 language by
reducing it to HTML 2.0:  markup in the form of a start-tag or end-tag
whose generic identifier is not declared is mapped to nothing during
tokenization.  Undeclared attributes are treated similarly.  The entire
attribute specification of an unknown attribute (i.e., the unknown
attribute and its value, if any) should be ignored.  On the other
hand, references to undeclared entities and undefined numeric
character references (i.e.  references to code positions that are not
in the domain of the document character set) should be treated as data
characters.
\par \par 
For example:
\par $<$div class=chapter$>$$<$h1$>$foo$<$/h1$>$$<$p$>$...$<$/div$>$
  =$>$ $<$H1$>$,"foo",$<$/H1$>$,$<$P$>$,"..."
xxx $<$P ID=z23$>$ yyy
  =$>$ "xxx ",$<$P$>$," yyy
Let \&alpha;, \&beta; \&\#97;nd \&\#76647; be finite sets.
  =$>$ "Let \&alpha;, \&beta; and \&76647; be finite sets."
\par 
Support for notifying the user of such errors is encouraged.
\par \par 
Information providers are warned that this convention is not binding:
unspecified behavior may result, as such markup is not conforming to
this specification.
\par \subsubsection*{Conventional Representation of Newlines}\par 
SGML specifies that a text entity is a sequence of records, each
beginning with a record start character and ending with a record end
character (code positions 10 and 13 respectively) (section
7.6.1, "Record Boundaries" in [SGML]).
\par \par 
[MIME] specifies that a body of type {\tt `text/*'} is a sequence of
lines, each terminated by CRLF, that is, octets 10, 13.
\par \par 
In practice, HTML documents are frequently represented and
transmitted using an end of line convention that depends on the
conventions of the source of the document;  frequently, that
representation consists of CR only, LF only, or a CR LF
sequence.  Hence the decoding of the octets will often result in
a text entity with some missing record start and record end
characters.
\par \par 
Since there is no ambiguity, HTML user agents are encouraged to
infer the missing record start and end characters.
\par \par 
An HTML user agent should treat end of line in any of its
variations as a word space in all contexts except
preformatted text.  Within preformatted text, an HTML user agent
should treat any of the three common representations of
end-of-line as starting a new line.
\par \subsection*{Security Considerations}\par 
Anchors, embedded images, and all other elements which contain
URIs as parameters may cause the URI to be dereferenced in response
to user input.  In this case, the security considerations of the URI
specification apply.
\par \par 
The widely deployed methods for submitting forms requests -- HTTP and
SMTP -- provide little assurance of confidentiality.  Information
providers who request sensitive information via forms -- especially by
way of the {\tt `PASSWORD'} type input field (see section Input Field:  INPUT)
-- should be aware and make their users aware of the lack of
confidentiality.
\par \section*{Document Structure}\par 
An HTML document is a hierarchy of elements.
\par \subsection*{Document Element:  HTML}\par 
The HTML document element consists of a head and a body, much
like a memo or a mail message.  The head contains the
title and other optional elements.  The body is a text flow
consisting of paragraphs, lists, and other elements.
\par \subsection*{Head:  HEAD}\par 
The head of an HTML document is an unordered collection of
information about the document.  For example:
\par $<$!DOCTYPE HTML PUBLIC "-//IETF//DTD HTML 2.0//EN"$>$
$<$HEAD$>$
$<$TITLE$>$Introduction to HTML$<$/TITLE$>$
$<$/HEAD$>$
...
\subsubsection*{Title:  TITLE}\par 
Every HTML document must contain a {\it TITLE} element.
\par \par 
The title should identify the contents of the document in a global
context.  A short title, such as "Introduction" may be meaningless out
of context.  A title such as "Introduction to HTML Elements" is more
appropriate.
(12)\par \par 
A user agent may display the title of a document in a history list
or as a label for the window displaying the document.  Contrast with
headings (section Headings:  H1 ...  H6), which are typically displayed with
the body text flow.
\par \subsubsection*{Base URI:  BASE}\par 
The optional {\it BASE} element specifies the URI of the document,
overriding any context otherwise known to the user agent.  The required
{\it HREF} attribute specifies the URI for navigating the document
(see section Hyperlinks).  The value of the {\it HREF} attribute
must be an absolute URI.
\par \subsubsection*{Keyword Index:  ISINDEX}\par 
The {\it ISINDEX} element indicates that the user agent should
allow the user to search an index by giving keywords.  See section Queries and Indexes for details.
\par \subsubsection*{Link:  LINK}\par 
The {\it LINK} element represents a hyperlink (see section Hyperlinks).  It has the same attributes as the {\it A} element
(see section Anchor:  A).
\par \par 
The {\it LINK} element is typically used to indicate authorship,
related indexes and glossaries, older or more recent versions,
stylesheets, document hierarchy etc.
\par \subsubsection*{Associated Metainformation:  META}\par 
The {\it META} element is an extensible container for use in
identifying, indexing, and cataloging specialized document
metainformation.  Metainformation has two main functions:
\par \begin{itemize}\item 
to provide a means to discover that the data set exists and
how it might be obtained or accessed;  and 
\item 
to document the content, quality, and features of a data set
and so give an indication of its fitness for use.  
\end{itemize}\par 
Each {\it META} element specifies a name/value pair.  If multiple
META elements are provided with the same name, their combined
contents--concatenated as a comma-separated list--is the value
associated with that name.
(13)\par \par 
HTTP servers should read the content of the document {\it HEAD} to
generate header fields corresponding to any elements defining a value
for the attribute {\it HTTP-EQUIV}.
(14)\par \par 
Attributes of the META element:
\par \begin{description}\item[{\it HTTP-EQUIV}]
This attribute binds the element to an HTTP header
field.  An HTTP server may use this information to process the
document.  In particular, it should include a header field in the
responses to {\it GET} requests for this document:  the header name is
taken from the {\it HTTP-EQUIV} attribute value, and the header value
is taken from the value of the {\it CONTENT} attribute.  HTTP header
names are not case sensitive.
\item[{\it NAME}]
name of the name/value pair.  If not present,
{\it HTTP-EQUIV} gives the name.
\item[{\it CONTENT}]
The value of the name/value pair.
\end{description}\par 
Examples
\par \par 
If the document contains:
\par $<$META HTTP-EQUIV="Expires"
      CONTENT="Tue, 04 Dec 1993 21:29:02 GMT"$>$
$<$meta http-equiv="Keywords" CONTENT="Fred, Barney"$>$
$<$META HTTP-EQUIV="Reply-to"
      content="fielding@ics.uci.edu (Roy Fielding)"$>$
\par 
then the server should include the following header fields:
\par Expires: Tue, 04 Dec 1993 21:29:02 GMT
Keywords: Fred, Barney
Reply-to: fielding@ics.uci.edu (Roy Fielding)
\par 
as part of the HTTP response to a {\tt `GET'} or {\tt `HEAD'} request
for that document.
\par \par 
When the HTTP-EQUIV attribute is not present, the server should not
generate an HTTP response header for the metainformation;  e.g.,
\par \par 
Do not name an HTTP-EQUIV equal to a response header that should
normally only be generated by the HTTP server.  Example names that are
inappropriate include {\tt `Server'}, {\tt `Date'}, and
{\tt `Last-modified'} -- the exact list of inappropriate names is
dependent on the particular server implementation.
\par \subsubsection*{Next Id:  NEXTID}\par 
They {\it NEXTID} element gives a hint for the name to use for an
{\it A} element when editing an HTML document.  It should be distinct
from all {\it NAME} attribute values on {\it A} elements.  For
example:
\par $<$NEXTID N=Z27$>$
\subsection*{Body:  BODY}\par 
The {\it BODY} element contains the text flow of the document,
including headings, paragraphs, lists, etc.
\par \par 
For example:
\par $<$BODY$>$
$<$h1$>$Important Stuff$<$/h1$>$
$<$p$>$Explanation about important stuff...
$<$/BODY$>$
\subsection*{Headings:  H1 ...  H6}\par 
The six heading elements, {\it H1} through {\it H6}, denote section
headings.  Although the order and occurence of headings is not
constrained by the HTML DTD, documents should not skip levels (for
example, from H1 to H3), as converting such documents to other
representations is often problematic.
\par \par 
Example of use:
\par $<$H1$>$This is a heading$<$/H1$>$
Here is some text
$<$H2$>$Second level heading$<$/H2$>$
Here is some more text.
\par 
Typical renderings are:
\par \begin{description}\item[{\it H1}]
Bold, very-large font, centered.  One or two blank lines
above and below.
\item[{\it H2}]
Bold, large font, flush-left.  One or two blank lines
above and below.
\item[{\it H3}]
Italic, large font, slightly indented from the left
margin.  One or two blank lines above and below.
\item[{\it H4}]
Bold, normal font, indented more than H3.  One blank
line above and below.
\item[{\it H5}]
Italic, normal font, indented as H4.  One blank line
above.
\item[{\it H6}]
Bold, indented same as normal text, more than H5.  One
blank line above.
\end{description}\subsection*{Block Structuring Elements}\par 
Each of the following elements defines a block structure;  that is,
they indicate a paragraph break before and after.
\par \subsubsection*{Paragraph:  P}\par 
The {\it P} element indicates a paragraph.  The exact indentation,
leading space, etc.  of a paragraph is not specified and may be a
function of other tags, style sheets, etc.
\par \par 
Typically, paragraphs are surrounded by a vertical space of one
line or half a line.  The first line in a paragraph is indented in some
cases.
\par \par 
Example of use:
\par $<$H1$>$This Heading Precedes the Paragraph$<$/H1$>$
$<$P$>$This is the text of the first paragraph.
$<$P$>$This is the text of the second paragraph. Although you do not 
need to start paragraphs on new lines, maintaining this 
convention facilitates document maintenance.$<$/P$>$
$<$P$>$This is the text of a third paragraph.$<$/P$>$
\subsubsection*{Preformatted Text:  PRE}\par 
The {\it PRE} element represents a character cell block of text and
so is suitable for text that has been formatted on screen.
\par \par 
The {\it PRE} tag may be used with the optional WIDTH attribute.  The
WIDTH attribute specifies the maximum number of characters for a line
and allows the HTML user agent to select a suitable font and
indentation.
\par \par 
Within preformatted text:
\par \begin{itemize}\item 
Line breaks within the text are rendered as a move to the
beginning of the next line.
(15)\item 
Anchor elements and phrase markup may be used.
(16)\item 
Elements that define paragraph formatting (headings, address,
etc.) must not be used.
(17)\item 
The horizontal tab character (encoded in {\tt `US-ASCII'} and
{\tt `ISO-8859-1'} as decimal 9) must be interpreted as the smallest
positive nonzero number of spaces which will leave the number of
characters so far on the line as a multiple of 8.
\end{itemize}\par 
Example of use:
\par $<$PRE$>$
Line 1.
       Line 2 is to the right of line 1.     $<$a href="abc"$>$abc$<$/a$>$
       Line 3 aligns with line 2.            $<$a href="def"$>$def$<$/a$>$
$<$/PRE$>$
\paragraph*{Example and Listing:  XMP, LISTING}\par 
The {\it XMP} and {\it LISTING} elements are deprectated in favor
of the {\it PRE} element (see section Preformatted Text:  PRE), as their declaration
makes use of {\tt `CDATA'} declared content, which tends to be
implemented and used inconsistently.
(18)\par \par 
The {\it LISTING} element should be rendered so that at least 132
characters fit on a line.  The {\it XMP} element should be rendered so
that at least 80 characters fit on a line but is otherwise identical
to the {\it LISTING} element.
(19)\par \subsubsection*{Address:  ADDRESS}\par 
The {\it ADDRESS} element specifies such information as address,
signature and authorship, often at the beginning or end of the body of
a document.
\par \par 
Typically, the {\it ADDRESS} element is rendered in an italic
typeface and may be indented.
\par \par 
Example of use:
\par $<$ADDRESS$>$
Newsletter editor$<$BR$>$
J.R. Brown$<$BR$>$
JimquickPost News, Jumquick, CT 01234$<$BR$>$
Tel (123) 456 7890
$<$/ADDRESS$>$
\subsubsection*{Block Quote:  BLOCKQUOTE}\par 
The {\it BLOCKQUOTE} element contains text quoted from another
source.
\par \par 
A typical rendering might be a slight extra left and right indent,
and/or italic font.  The {\it BLOCKQUOTE} typically provides space
above and below the quote.
\par \par 
Single-font rendition may reflect the quotation style of Internet
mail by putting a vertical line of graphic characters, such as the
greater than symbol ($>$), in the left margin.
\par \par 
Example of use:
\par I think the poem ends
$<$BLOCKQUOTE$>$
$<$P$>$Soft you now, the fair Ophelia. Nymph, in thy orisons, be all 
my sins remembered.
$<$/BLOCKQUOTE$>$
but I am not sure.
\subsection*{List Elements}\par 
HTML includes a number of list elements.  They may be used in
combination;  for example, a {\it OL} may be nested in an {\it LI}
element of a {\it UL}.
\par \subsubsection*{Unordered List:  UL, LI}\par 
The {\it UL} represents a list of items with no inherent ordering
-- typically a bulleted list.
\par \par 
The content of a {\it UL} element is a sequence of {\it LI}
elements.  For example:
\par $<$UL$>$
$<$LI$>$First list item
$<$LI$>$Second list item
 $<$p$>$second paragraph of second item
$<$LI$>$Third list item
$<$/UL$>$
\subsubsection*{Ordered List:  OL}\par 
The {\it UL} element represents an ordered list of items, sorted by
sequence or order of importance.
\par \par 
The content of a {\it OL} element is a sequence of {\it LI}
elements.  For example:
\par $<$OL$>$
$<$LI$>$Click the Web button to open the Open the URI window.
$<$LI$>$Enter the URI number in the text field of the Open URI 
window. The Web document you specified is displayed.
  $<$ol$>$
   $<$li$>$substep 1
   $<$li$>$substep 2
  $<$/ol$>$
$<$LI$>$Click highlighted text to move from one link to another.
$<$/OL$>$
\par 
The {\it COMPACT} attribute suggests that a compact rendering be
used.
\par \subsubsection*{Directory List:  DIR}\par 
The {\it DIR} element is similar to the {\it UL} element.  It
represents a list of short items, typically up to 20 characters
each.  Items in a directory list may be arranged in columns, typically
24 characters wide.
\par \par 
The content of a {\it OL} element is a sequence of {\it LI}
elements.  Nested block elements are not allowed in the content of
{\it DIR} elements.  For example:
\par $<$DIR$>$
$<$LI$>$A-H$<$LI$>$I-M
$<$LI$>$M-R$<$LI$>$S-Z
$<$/DIR$>$
\subsubsection*{Menu List:  MENU}\par 
The {\it MENU} element is a list of items with typically one line
per item.  The menu list style is typically more compact than the style
of an unordered list.
\par \par 
The content of a {\it MENU} element is a sequence of {\it LI}
elements.  Nested block elements are not allowed in the content of
{\it MENU} elements.  For example:
\par $<$MENU$>$
$<$LI$>$First item in the list.
$<$LI$>$Second item in the list.
$<$LI$>$Third item in the list.
$<$/MENU$>$
\subsubsection*{Definition List:  DL, DT, DD}\par 
A definition list is a list of terms and corresponding
definitions.  Definition lists are typically formatted with the term
flush-left and the definition, formatted paragraph style, indented
after the term.
\par \par 
The content of a DL element is a sequence of DT elements and/or DD
elements, usually in pairs.
\par \par 
Example of use:
\par $<$DL$>$
$<$DT$>$Term$<$DD$>$This is the definition of the first term.
$<$DT$>$Term$<$DD$>$This is the definition of the second term.
$<$/DL$>$
\par 
If the DT term does not fit in the DT column (one third of the
display area), it may be extended across the page with the DD section
moved to the next line, or it may be wrapped onto successive lines of
the left hand column.
\par \par 
The optional {\it COMPACT} attribute suggests that a compact
rendering be used, because the list items are small and/or the entire
list is large.
\par \par 
Unless the {\it COMPACT} attribute is present, an HTML user agent
may leave white space between successive DT, DD pairs.  The
{\it COMPACT} attribute may also reduce the width of the left-hand
(DT) column.
\par $<$DL COMPACT$>$
$<$DT$>$Term$<$DD$>$This is the first definition in compact format.
$<$DT$>$Term$<$DD$>$This is the second definition in compact format.
$<$/DL$>$
\subsection*{Phrase Markup}\par 
Phrases may be marked up according to idiomatic usage,
typographic appearance, or for use as hyperlink anchors.
\par \par 
User agents must render highlighted phrases distinctly from plain
text.  Additionally, {\it EM} content must be rendered as distinct from
{\it STRONG} content, and {\it B} content must rendered as distinct
from {\it I} content.
\par \par 
Phrase elements may be nested within the content of other phrase
elements;  however, HTML user agents may render nested phrase
elements indistinctly from non-nested elements:
\par plain $<$B$>$bold $<$I$>$italic$<$/I$>$$<$/B$>$ may the rendered 
the same as plain $<$B$>$bold $<$/B$>$$<$I$>$italic$<$/I$>$
\subsubsection*{Idiomatic Elements}\par 
Phrases may be marked up to indicate certain idioms.
(20)\par \paragraph*{Citation:  CITE}\par 
The {\it CITE} element is used to indicate the title of a book or
other citation.  It is typically rendered as italics.  For example:
\par He just couldn't get enough of $<$cite$>$The Grapes of Wrath$<$/cite$>$.
\paragraph*{Code:  CODE}\par 
The {\it CODE} element indicates an example of code, typically
rendered in a monospaced font.  Contrast with the {\it PRE} block
structuring element in section Preformatted Text:  PRE.  For example:
\par The expression $<$code$>$x += 1$<$/code$>$
is short for $<$code$>$x = x + 1$<$/code$>$.
\paragraph*{Emphasis:  EM}\par 
The {\it EM} element indicates an emphasized phrase, typically
rendered as italics.  For example:
\par A singular subject $<$em$>$always$<$/em$>$ takes a singular verb.
\paragraph*{Keyboard:  KBD}\par 
The Keyboard element indicates text typed by a user, typically
rendered in a monospaced font.  This is commonly used in instruction
manuals.  For example:
\par Enter $<$kbd$>$FIND IT$<$/kbd$>$ to search the database.
\paragraph*{Sample:  SAMP}\par 
The {\it SAMP} element indicates a sequence of literal characters,
typically rendered in a monospaced font.  For example:
\par The only word containing the letters $<$samp$>$mt$<$/samp$>$ is dreamt.
\paragraph*{Strong Empasis:  STRONG}\par 
The {\it STRONG} element indicates strong emphasis,
typically rendered in bold.  For example:
\par $<$strong$>$STOP$<$/strong$>$, or I'll say "$<$strong$>$STOP$<$/strong$>$" again!.
\paragraph*{Variable:  VAR}\par 
The {\it VAR} element indicates a placeholder, typically rendered
as italic.  For example:
\par Take a guess: Roses are $<$var$>$blank$<$/var$>$.
\subsubsection*{Typographic Elements}\par 
Typographic elements are used to specify the format of marked text.
\par \par 
Typical renderings for idomatic elements vary between user
agents.  If a specific rendering is necessary -- for example, when
referring to a specific text attribute as in "The italic parts are
mandatory" -- a typographic element can be used to ensure that the
intended typography is used where possible.
\par \par (21)\par \paragraph*{Bold:  B}\par 
The {\it B} element indicates bold text.  Where bold typography is
unavailable, an alternative representation may be used.
\par \paragraph*{Italic:  I}\par 
The {\it I} element indicates italic text.  Where italic typography
is unavailable, an alternative representation may be used.
\par \paragraph*{Typewriter:  TT}\par 
The {\it TT} element indicates typewriter text.  Where a typewriter
font is unavailable, an alternative representation may be used.
\par \subsubsection*{Anchor:  A}\par 
The {\it A} element indicates a hyperlink anchor (see section Hyperlinks).  At least one of the {\it NAME} and {\it HREF}
attributes should be present.  Attributes of the {\it A} element:
\par \begin{description}\item[{\it HREF}]
gives the URI of the head anchor of a hyperlink.
\item[{\it NAME}]
gives the name of the anchor, and makes it available as
a head of a hyperlink.
\item[{\it TITLE}]
suggests a title for the destination resource -- advisory only.  
The {\it TITLE} attribute may be used:
\begin{itemize}\item 
for display prior to accessing the destination resource, for
example, as a margin note or on a small box while the mouse is over
the anchor, or while the document is being loaded;
\item 
for resources that do not specify a title such as graphics, plain
text and Gopher menus, for use as a window title.
\end{itemize}\item[{\it REL}]
The {\it REL} attribute gives the relationship(s) described
by the hyperlink.  The value is a whitespace separated list of
relationship names.
\item[{\it REV}]
same as the REL attribute, but the semantics of the
relationship are in the reverse direction.  A link from A to B with
REL="X" expresses the same relationship as a link from B to A with
REV="X".  An anchor may have both REL and REV attributes.
\item[{\it URN}]
specifies a preferred, more persistent identifier for the
head anchor of the hyperlink.  The format of URNs is under discussion
(1995) by various working groups of the Internet Engineering Task
Force.
\item[{\it METHODS}]
specifies methods to be used in accessing the
destination, as a whitespace-separated list of names.  For similar
reasons as for the TITLE attribute, it may be useful to include the
information in advance in the link.  For example, the HTML user agent
may chose a different rendering as a function of the methods allowed;
for example, something that is searchable may get a different icon.
\end{description}\subsection*{Line Break:  BR}\par 
The {\it BR} element specifies a line break between words (see
section Characters, Words, and Paragraphs).  For example:
\par $<$P$>$ Pease porridge hot$<$BR$>$
Pease porridge cold$<$BR$>$
Pease porridge in the pot$<$BR$>$
Nine days old.
\subsection*{Horizontal Rule:  HR}\par 
The {\it HR} element is a divider between sections of text;
typcially a full width horizontal rule or equivalent graphic.  For
example:
\par $<$HR$>$
$<$ADDRESS$>$February 8, 1995, CERN$<$/ADDRESS$>$
$<$/BODY$>$
\subsection*{Image:  IMG}\par 
The {\it IMG} element refers to an image or icon via a hyperlink
(see section Simultaneous Presentation of Image Resources).
\par \par 
HTML user agents may process the value of the {\it ALT} attribute
as an alternative to processing the image resource indicated by the
{\it SRC} attribute.
(22)\par \par 
Attributes of the {\it IMG} element:
\par \begin{description}\item[{\it ALIGN}]
alignment of the image with respect to the text
baseline.
\begin{itemize}\item {\tt `TOP'} specifies that the top of the image aligns with the
tallest item on the line contianing the image.
\item {\tt `MIDDLE'} specifies that the center of the image aligns with
the baseline of the line containing the image.
\item {\tt `BOTTOM'} specifies that the bottom of the image aligns with
the baseline of the line containing the image.
\end{itemize}\item[{\it ALT}]
text to use in place of the referenced image resource, for
example due to processing constraints or user preference.
\item[{\it ISMAP}]
indicates an image map (see section Image Maps).
\item[{\it SRC}]
specifies the URI of the image resource.
(23)\end{description}\par 
Examples of use:
\par $<$IMG SRC="triangle.xbm" ALT="Warning:"$>$ Be sure 
to read these instructions.
$<$IMG SRC="triangle.xbm"$>$Be sure to read these 
instructions.
$<$a href="http://machine/htbin/imagemap/sample"$>$
$<$IMG SRC="sample.xbm" ISMAP$>$
$<$/a$>$
\section*{Characters, Words, and Paragraphs}\par 
An HTML user agent should present the body of an HTML document as
a collection of typeset paragraphs and preformatted text.  Except
for the {\it PRE} element, each block structuring element is regarded as
a paragraph by taking the data characters in its content and the
content of its descendant elements, concatenating them, and
splitting the result into words, separated by space, tab, or
record end characters (and perhaps hyphen characters).  The
sequence of words is typeset as a paragraph by breaking it into
lines.
\par \subsection*{The ISO Latin 1 Character Repertoire}\par 
The minimum character repertoire supported by all conforming HTML user
agents is Latin Alphabet No.  1, or simply Latin-1.  Latin-1 includes
characters from most Western European languages, as well as a number
of control characters.  Latin-1 also includes a non-breaking space, a
soft hyphen indicator, 93 graphical characters, 8 unassigned
characters, and 25 control characters.
(24)(25)\par \par 
In SGML applications, the use of control characters is limited in
order to maximize the chance of successful interchange over
heterogeneous networks and operating systems.  In HTML, only three
control characters are allowed:  Horizontal Tab, Carriage Return, and
Line Feed (code positions 9, 13, and 10 in ISO 8859-1).
\par \par 
The HTML DTD references the Added Latin 1 entity set, to allow
mnemonic representation of selected Latin 1 characters using only the
widely supported ASCII character repertoire.  For example:
\par Kurt G\&ouml;del was a famous logician and mathematician.
\par 
See section ISO Latin 1 Character Entity Set for a table of the "Added Latin 1" entities,
and section The ANSI/ISO 8859-1 Coded Character Set for a table of the code positions of
ANSI/ISO 8859-1.
\par \section*{Hyperlinks}\par 
In addition to general purpose elements such as paragraphs and
lists, HTML documents can express hyperlinks.  A hyperlink is a
relationship between two anchors, called the head and the tail of the
hyperlink[DEXTER].  An anchor is a resource such as an HTML document,
or some fragment of, i.e.  view on or portion of a resource.
\par \par 
Anchors are addressed by Uniform Resource Identifiers (URI).  URIs
either refer directly to an anchor in absolute form for example as in
[URL], or they refer to an anchor relative to a base URI which is
absolute, as in [RELURL].
\par \par 
Each of the following markup constructs indicates the tail anchor
of a hyperlink or set of hyperlinks:
\par \begin{itemize}\item {\it A} elements with {\it HREF} present.
\item {\it LINK} elements.
\item {\it IMG} elements.
\item {\it INPUT} elements with the {\it SRC} attribute present.
\item {\it ISINDEX} elements.
\item {\it FORM} elements with {\tt `METHOD=GET'}.
\end{itemize}\subsection*{Accessing Resources}\par 
To access the head anchor of a hyperlink, the user agent
determines its URI from the URI given in the tail anchor, using the
base URI of the document containing the head anchor if necessary.  Any
fragment identifier is discarded, and the result is used to access a
resource, for example as in [URL].
\par \par 
For example, if a document identified as {\tt `http://host/x/y.html'}
contains:
\par $<$img src="../icons/abc.gif"$>$
\par 
then the user agent must use the URI
{\tt `http://host/icons/abc.gif'} to access the resource linked from
the {\it IMG} element.
\par \subsection*{Activation of Hyperlinks}\par 
An HTML user agent allows the user to navigate the content of the
document and request activation of {\it A} element hyperlinks.  A
request to activate a link is essentially a request to process the
resource indicated by the head anchor of the link, for example to
display the indicated HTML document.  HTML user agents should also
allow activation of {\it LINK} element hyperlinks.
\par \par 
The base URI for navigating the head anchor may be different from
the URI used to access it.  For example, it may be replaced by by a
{\it BASE} tag in the destination document or by an HTTP redirection
transaction.
\par \subsection*{Simultaneous Presentation of Image Resources}\par 
An HTML user agent may activate hyperlinks indicated by {\it IMG} and
{\it INPUT} elements concurrently with processing the document;  that
is, image hyperlinks may be processed without explicit request by the
user.  Image resources should be embedded in the presentation at the
point of the tail anchor, that is the {\it IMG} element.
\par \par {\it LINK} hyperlinks may also be processed without explicit user
request;  for example, style sheet resources may be processed before or
during the processing of the document.
\par \subsubsection*{Fragment Identifiers}\par 
Any characters following a {\tt `\#'} character in a URI constitute a
fragment identifier.  As a degenerate case, a URI of the form
{\tt `\#fragment'} refers to an anchor in the same document.
\par \par 
The meaning of fragment identifiers depends on the media type of the
resource containing the head anchor.  For {\tt `text/html'} resources,
it refers to the {\it A} element with a {\it NAME} attribute whose
value is the same as the fragment identifier.  The matching is case
sensitive.  The document should have exactly one such element.  The user
agent should indicate the anchor element, for example by scrolling to
and/or highlighting the phrase.
\par \par 
For example, if a user agent was processing a document identified as
{\tt `http://host/x/y.html'} and the user indicated the following
anchor:
\par $<$p$>$ See: $<$a href="app1.html\#bananas"$>$appendix 1$<$/a$>$ 
for more detail on bananas.
\par 
then the user agent URI must access the resource
{\tt `http://host/x/app1.html'}.  Assuming the resource is represented
using the {\tt `text/html'} media type, the user agent must locate
the anchor named {\tt `bananas'} and begin navigation there.
\par \subsection*{Queries and Indexes}\par 
The {\it ISINDEX} element represents a set of hyperlinks.  The user
can choose from the set by providing keywords to the user agent.  The
user agent computes the head URI by appending {\tt `?'} and the
keywords to the base URI.  The keywords are escaped according to [URL]
and joined by {\tt `+'}.  For example, if a document contains:
\par $<$BASE HREF="http://host/index"$>$
$<$ISINDEX$>$
\par 
and the user provides the keywords {\tt `apple'} and {\tt `berry'},
then the user agent must access the resource
{\tt `http://host/index?apple+berry'}.
\par \par {\it FORM} elements with {\tt `METHOD=GET'} also represent sets of
hyperlinks.  See section Query Forms:  METHOD=GET for details.
\par \subsection*{Image Maps}\par 
The {\it ISMAP} attribute in combination with the {\it A} and
{\it IMG} elements, represents a set of hyperlinks.  The user can
choose from the set by choosing a pixel of the image.  The user agent
computes the head URI by appending {\tt `?'} and the x and y
coordinates of the pixel to the URI given in the {\it A} element.  For
example, if a document contains:
\par $<$!DOCTYPE HTML PUBLIC "-//IETF//DTD HTML 2.0//EN"$>$
$<$head$>$$<$title$>$ImageMap Example$<$/title$>$
$<$BASE HREF="http://host/index"$>$$<$/head$>$
$<$body$>$
$<$p$>$ Choose any of these icons:$<$br$>$
$<$a href="/cgi-bin/imagemap"$>$$<$img ismap src="icons.gif"$>$$<$/a$>$
\par 
and the user chooses the upper-leftmost pixel, then chosen
hyperlink is the one with the URI
{\tt `http://host/cgi-bin/imagemap?0,0'}.
\par \section*{Forms}\par 
A form is a template for a form data set and an associated method
and action URI.  A form data set is a sequence of name/value pair
fields.  The names are specified on the {\it NAME} attributes of form
input elements, and the values are given initial values by various
forms of markup and edited by the user.  The resulting form data set is
used to access an information service as a function of the action and
method.
\par \par 
Forms elements can be mixed in with document structuring
elements.  For example, a {\it PRE} element may contain a {\it FORM}
element, or a {\it FORM} element may contain lists which contain
{\it INPUT} elements.  This gives considerable flexibility in
designing the layout of forms.
\par \par 
Form processing is a level 2 feature.
\par \subsection*{Form Elements}\subsubsection*{Form:  FORM}\par 
The {\it FORM} element contains a sequence of input elements, along
with document structuring elements.  The attributes are:
\par \begin{description}\item[{\it ACTION}]
specifies the action URI for the form.  The {\it ACTION}
attribute defaults to the base URI of the document (see section Hyperlinks).
\item[{\it METHOD}]
selects a method of accessing the action URI.
\item[{\it ENCTYPE}]
specifies the media type used to encode the name/value
pairs for transport, in case the protocol does not itself impose a
format.
\end{description}\subsubsection*{Input Field:  INPUT}\par 
The {\it INPUT} element represents a field for user input.  Attributes are:
\par \begin{description}\item[{\it ALIGN}]
vertical alignment of the image.  For use only with
{\tt `TYPE=IMAGE'}.  The possible values are as for the {\it ALIGN}
attribute of the {\it IMG} element (see section Image:  IMG).
\item[{\it CHECKED}]
indicates that the initial state of a checkbox or radio
button is selected.
\item[{\it MAXLENGTH}]
constrains the number of characters that can be
entered into a text input field.  If the value of {\it MAXLENGTH} is
greater the the value of the {\it SIZE} attribute, the field should
scroll appropriately.  The default number of characters is unlimited.
\item[{\it NAME}]
symbolic name for the form field corresponding to this
element or group of elements.
\item[{\it SIZE}]
specifies the amount of display space allocated to this
input field according to its type.
\item[{\it SRC}]
A URI specifying an image resource.  For use only with
{\tt `TYPE=IMAGE'}.
\item[{\it TYPE}]
indicates type of the field.  Defaults to
{\tt `TEXT'}.  Values are:
\begin{description}\item[{\it CHECKBOX}]
an independent boolean value.
\item[{\it HIDDEN}]
a hidden field.  The user does not interact with this
field;  instead, the {\it VALUE} attribute can be used to specify a
value.
\item[{\it IMAGE}]
specifies an image resource to display, and allows input
of two form data:  the x and y coordinate of a pixel chosen from the
image.  The names of the data are the name of this element with
{\tt `.x'} and {\tt `.y'} appended.  {\tt `TYPE=IMAGE'} implies
{\tt `TYPE=SUBMIT'} processing;  that is, when a pixel is chosen, the
form as a whole is submitted.
\item[{\it PASSWORD}]
Similar to the TEXT attribute, except that the value is
obscured as it is entered.
\item[{\it RADIO}]
a 1-of-many choice.  All {\it INPUT} elements with
{\tt `TYPE=RADIO'} and the same {\it NAME} combine into one form
field.  The value of the form field is the {\it VALUE} of the element
chosen by the user.  The initial state may be indicated with the
{\it CHECKED} attribute.  The {\it VALUE} attribute is required for
radio inputs.
\end{description}\item[{\it RESET}]
an input option, typically a button, that instructs the
user agent to reset the form's fields to their initial states.  Any
{\it VALUE} attribute indicates a label for the input (button).
\item[{\it SUBMIT}]
an input option, typically a button, that instructs the
user agent to submit the form.  Any {\it VALUE} attribute indicates a
label for the input (button).  If the {\it NAME} attribute is present,
this element contributes a form field whose value is given by the
{\it VALUE} attribute.  If the {\it NAME} attribute is not present,
this element does not contribute a form field.
\item[{\it TEXT}]
a single line text entry fields.  The {\it SIZE} and
{\it MAXLENGTH} attributes may be used to constrain the input or
layout of the field.  Use the {\it TEXTAREA} element for multi-line
text fields.
\item[{\it VALUE}]
The initial value of the field.
\end{description}\subsubsection*{Selection:  SELECT}\par 
The {\it SELECT} element constrains the form field to an enumerated
list of values.  The values are given in {\it OPTION} elements.
Attributes are:
\par \begin{description}\item[{\it MULTIPLE}]
indicates that more than one option may be included in
the value.
\item[{\it NAME}]
specifies the name of the form field.
\item[{\it SIZE}]
specifies the number of visible items.  Select fields of
size one are typically pop-down menus, whereas select fields with size
greater than one are typically lists.
\end{description}\par 
For example:
\par $<$SELECT NAME="flavor"$>$
$<$OPTION$>$Vanilla
$<$OPTION$>$Strawberry
$<$OPTION$>$Rum and Raisin
$<$OPTION$>$Peach and Orange
$<$/SELECT$>$
\par 
The initial state has the first option selected, unless a
{\it SELECTED} attribute is present on any of the {\it OPTION}
elements.
\par \paragraph*{Option:  OPTION}\par 
The Option element can only occur within a Select element.  It
represents one choice, and has the following attributes:
\par \begin{description}\item[{\it SELECTED}]
Indicates that this option is initially selected.
\item[{\it VALUE}]
indicates the value to be returned if this
option is chosen.  The field value defaults to the content of the
{\it OPTION} element.
\end{description}\par 
The content of the {\it OPTION} element is presented to the user to
represent the option.  It is used as a returned value if the VALUE
attribute is not present.
\par \subsubsection*{Text Area:  TEXTAREA}\par 
The {\it TEXTAREA} element represents a multi-line text
field.  Attrubutes are:
\par \begin{description}\item[{\it COLS}]
the number of visible columns to display for the text area,
in characters.
\item[{\it NAME}]
Specifies the name of the form field.
\item[{\it ROWS}]
The number of visible rows to display for the text area, in
characters.
\end{description}\par 
For example:
\par $<$TEXTAREA NAME="address" ROWS=64 COLS=6$>$
HaL Computer Systems
1315 Dell Avenue
Campbell, California 95008
$<$/TEXTAREA$>$
\par 
The content of the {\it TEXTAREA} element is the field's initial
value.
\par \par 
Typically, the ROWS and COLS attributes determine the visible
dimension of the field in characters.  The field is typically rendered
in a fixed-width font.  HTML user agents should allow text to extend
beyond these limits by scrolling as needed.
\par \subsection*{Form Submission}\par 
An HTML user agent begins processing a form by presenting the
document with the fields in their initial state.  The user is allowed
to modify the fields, constrained by the field type etc.  When the user
indicates that the form should be submitted (using a submit button or
image input), the form data set is processed according to its method,
action URI and enctype.
\par \par 
When there is only one single-line text input field in a form, the
user agent should accept Enter in that field as a request to submit
the form.
\par \subsubsection*{The form-urlencoded Media Type}\par 
The default encoding for all forms is
{\tt `application/x-www-form-urlencoded'}.  A form data set is
represented in this media type as follows:
\par \begin{enumerate}\item 
The form field names and values are escaped:  space characters are
replaced by {\tt `+'}, and then reserved characters are escaped as per
[URL];  that is, non-alphanumeric characters are replaced by
{\tt `\%HH'}, a percent sign and two hexadecimal digits representing the
ASCII code of the character.  Line breaks, as in multi-line textfield
values, are represented as CR LF pairs, i.e.  {\tt `\%0D\%0A'}.
\item 
The fields are listed in the order they appear in the document
with the name separated from the value by {\tt `='} and the pairs
separated from each other by {\tt `\&'}.  Fields with null values may be
omitted.  In particular, unselected radio buttons and checkboxes should
not appear in the encoded data, but hidden fields with {\it VALUE}
attributes present should.
(26)\end{enumerate}\subsubsection*{Query Forms:  METHOD=GET}\par 
If the processing of a form is idempotent (i.e.  it has no lasting
observable effect on the state of the world), then the form method
should be {\tt `GET'}.  Many database searches have no visible
side-effects and make ideal applications of query forms.
\par \par 
To process a form whose action URL is an HTTP URL and whose method
is {\tt `GET'}, the user agent starts with the action URI and appends a
{\tt `?'} and the form data set, in
{\tt `application/x-www-form-urlencoded'} format as above.  The user
agent then traverses the link to this URI just as if it were an anchor
(see section Activation of Hyperlinks).
(27)\par \subsubsection*{Forms with Side-Effects:  METHOD=POST}\par 
If the service associated with the processing of a form has side
effects (for example, modification of a database or subscription to a
service), the method should be {\tt `POST'}.
\par \par 
To process a form whose action URL is an HTTP URL and whose method
is {\tt `POST'}, the user agent conducts an HTTP POST transaction using
the action URI, and a message body of type
{\tt `application/x-www-form-urlencoded'} format as above.  The user
agent should display the response from the HTTP POST interaction just
as it would display the response from an HTTP GET above.
\par \subsubsection*{Example Form Submission:  Questionnaire Form}\par 
Consider the following document:
\par $<$!DOCTYPE HTML PUBLIC "-//IETF//DTD HTML 2.0//EN"$>$
$<$title$>$Sample of HTML Form Submission$<$/title$>$
$<$H1$>$Sample Questionnaire$<$/H1$>$
$<$P$>$Please fill out this questionnaire:
$<$FORM METHOD="POST" ACTION="http://www.w3.org/sample"$>$
$<$P$>$Your name: $<$INPUT NAME="name" size="48"$>$
$<$P$>$Male $<$INPUT NAME="gender" TYPE=RADIO VALUE="male"$>$
$<$P$>$Female $<$INPUT NAME="gender" TYPE=RADIO VALUE="female"$>$
$<$P$>$Number in family: $<$INPUT NAME="family" TYPE=text$>$
$<$P$>$Cities in which you maintain a residence:
$<$UL$>$
$<$LI$>$Kent $<$INPUT NAME="city" TYPE=checkbox VALUE="kent"$>$
$<$LI$>$Miami $<$INPUT NAME="city" TYPE=checkbox VALUE="miami"$>$
$<$LI$>$Other $<$TEXTAREA NAME="other" cols=48 rows=4$>$$<$/textarea$>$
$<$/UL$>$
Nickname: $<$INPUT NAME="nickname" SIZE="42"$>$
$<$P$>$Thank you for responding to this questionnaire.
$<$P$>$$<$INPUT TYPE=SUBMIT$>$ $<$INPUT TYPE=RESET$>$
$<$/FORM$>$
\par 
The inital state of the form data set is:
\par \begin{description}\item[{\it name}]
""
\item[{\it gender}]
"male"
\item[{\it family}]
""
\item[{\it other}]
""
\item[{\it nickname}]
""
\end{description}\par 
Note that the radio input has an initial value, while the checkbox
has none.
\par \par 
The user might edit the fields and request that the form be
submitted.  At that point, suppose the values are:
\par \begin{description}\item[{\it name}]
"John Doe"
\item[{\it gender}]
"male"
\item[{\it family}]
"5"
\item[{\it city}]
"kent,miami"
\item[{\it other}]
"abc$\backslash$ndef"
\item[{\it nickname}]
"J\&D"
\end{description}\par 
The user agent then conducts an HTTP POST transaction using the URI
{\tt `http://www.w3.org/sample'}.  The message body would be (ignore
the linebreak):
\par name=John+Doe\&gender=male\&family=5\&city=kent\%2Cmiami\&
other=abc\%0D\%0Adef\&nickname=J\%26D
\section*{HTML Public Text}\subsection*{HTML DTD}\par 
This is the Document Type Definition for the HyperText Markup
Language, level 2.
\par $<$!--    html.dtd
        Document Type Definition for the HyperText Markup Language               
 (HTML DTD)        
\$Id$        
Author: Daniel W. Connolly $<$connolly@w3.org$>$        
See Also: html.decl, html-1.dtd        
  http://www.w3.org/hypertext/WWW/MarkUp/MarkUp.html
--$>$
$<$!ENTITY \% HTML.Version
        "-//IETF//DTD HTML 2.0//EN"
        -- Typical usage:
            $<$!DOCTYPE HTML PUBLIC "-//IETF//DTD HTML//EN"$>$
            $<$html$>$
            ...
            $<$/html$>$
        --
        $>$
$<$!--============ Feature Test Entities ========================--$>$
$<$!ENTITY \% HTML.Recommended "IGNORE"        
-- Certain features of the language are necessary for        
   compatibility with widespread usage, but they may        
   compromise the structural integrity of a document.        
   This feature test entity enables a more prescriptive        
   document type definition that eliminates        
   those features.        
--$>$
$<$![ \%HTML.Recommended [
        $<$!ENTITY \% HTML.Deprecated "IGNORE"$>$
]]$>$
$<$!ENTITY \% HTML.Deprecated "INCLUDE"        
-- Certain features of the language are necessary for        
   compatibility with earlier versions of the specification,        
   but they tend to be used an implemented inconsistently,        
   and their use is deprecated. This feature test entity        
   enables a document type definition that eliminates        
   these features.        
--$>$
$<$!ENTITY \% HTML.Highlighting "INCLUDE"        
-- Use this feature test entity to validate that a        
   document uses no highlighting tags, which may be        
   ignored on minimal implementations.        
--$>$
$<$!ENTITY \% HTML.Forms "INCLUDE"
        -- Use this feature test entity to validate that a document
           contains no forms, which may not be supported in minimal
           implementations
        --$>$
$<$!--============== Imported Names ==============================--$>$
$<$!ENTITY \% Content-Type "CDATA"
        -- meaning an internet media type
           (aka MIME content type, as per RFC1521)
        --$>$
$<$!ENTITY \% HTTP-Method "GET | POST"
        -- as per HTTP specification, in progress
        --$>$
$<$!ENTITY \% URI "CDATA"
        -- The term URI means a CDATA attribute
           whose value is a Uniform Resource Identifier,
           as defined by         
"Universal Resource Identifiers" by Tim Berners-Lee        
aka RFC 1630        
Note that CDATA attributes are limited by the LITLEN        
capacity (1024 in the current version of html.decl),        
so that URIs in HTML have a bounded length.
        --$>$
$<$!--========= DTD "Macros" =====================--$>$
$<$!ENTITY \% heading "H1|H2|H3|H4|H5|H6"$>$
$<$!ENTITY \% list " UL | OL | DIR | MENU " $>$
$<$!--======= Character mnemonic entities =================--$>$
$<$!ENTITY \% ISOlat1 PUBLIC
  "ISO 8879-1986//ENTITIES Added Latin 1//EN//HTML"$>$
\%ISOlat1;
$<$!ENTITY amp CDATA "\&\#38;"     -- ampersand          --$>$
$<$!ENTITY gt CDATA "\&\#62;"      -- greater than       --$>$
$<$!ENTITY lt CDATA "\&\#60;"      -- less than          --$>$
$<$!ENTITY quot CDATA "\&\#34;"    -- double quote       --$>$
$<$!--========= SGML Document Access (SDA) Parameter Entities =====--$>$
$<$!-- HTML 2.0 contains SGML Document Access (SDA) fixed attributes
in support of easy transformation to the International Committee
for Accessible Document Design (ICADD) DTD        
 "-//EC-USA-CDA/ICADD//DTD ICADD22//EN".
ICADD applications are designed to support usable access to
structured information by print-impaired individuals through
Braille, large print and voice synthesis.  For more information on
SDA \& ICADD:  
        - ISO 12083:1993, Annex A.8, Facilities for Braille,        
  large print and computer voice
        - ICADD ListServ        
  $<$ICADD\%ASUACAD.BITNET@ARIZVM1.ccit.arizona.edu$>$
        - Usenet news group bit.listserv.easi
        - Recording for the Blind, +1 800 221 4792
--$>$
$<$!ENTITY \% SDAFORM  "SDAFORM  CDATA  \#FIXED"        
  -- one to one mapping        --$>$
$<$!ENTITY \% SDARULE  "SDARULE  CDATA  \#FIXED"        
  -- context-sensitive mapping --$>$
$<$!ENTITY \% SDAPREF  "SDAPREF  CDATA  \#FIXED"        
  -- generated text prefix     --$>$
$<$!ENTITY \% SDASUFF  "SDASUFF  CDATA  \#FIXED"        
  -- generated text suffix     --$>$
$<$!ENTITY \% SDASUSP  "SDASUSP  NAME   \#FIXED"        
  -- suspend transform process --$>$
$<$!--========== Text Markup =====================--$>$
$<$![ \%HTML.Highlighting [
$<$!ENTITY \% font " TT | B | I "$>$
$<$!ENTITY \% phrase "EM | STRONG | CODE | SAMP | KBD | VAR | CITE "$>$
$<$!ENTITY \% text "\#PCDATA | A | IMG | BR | \%phrase | \%font"$>$
$<$!ELEMENT (\%font;|\%phrase) - - (\%text)*$>$
$<$!ATTLIST ( TT | CODE | SAMP | KBD | VAR )
        \%SDAFORM; "Lit"
        $>$
$<$!ATTLIST ( B | STRONG )
        \%SDAFORM; "B"
        $>$
$<$!ATTLIST ( I | EM | CITE )
        \%SDAFORM; "It"
        $>$
$<$!-- $<$TT$>$       Typewriter text                         --$>$
$<$!-- $<$B$>$        Bold text                               --$>$
$<$!-- $<$I$>$        Italic text                             --$>$
$<$!-- $<$EM$>$       Emphasized phrase                       --$>$
$<$!-- $<$STRONG$>$   Strong emphais                          --$>$
$<$!-- $<$CODE$>$     Source code phrase                      --$>$
$<$!-- $<$SAMP$>$     Sample text or characters               --$>$
$<$!-- $<$KBD$>$      Keyboard phrase, e.g. user input        --$>$
$<$!-- $<$VAR$>$      Variable phrase or substituable         --$>$
$<$!-- $<$CITE$>$     Name or title of cited work             --$>$
$<$!ENTITY \% pre.content "\#PCDATA | A | HR | BR | \%font | \%phrase"$>$
]]$>$
$<$!ENTITY \% text "\#PCDATA | A | IMG | BR"$>$
$<$!ELEMENT BR    - O EMPTY$>$
$<$!ATTLIST BR
        \%SDAPREF; "\&\#RE;"
        $>$
$<$!-- $<$BR$>$       Line break      --$>$
$<$!--========= Link Markup ======================--$>$
$<$![ \%HTML.Recommended [
        $<$!ENTITY \% linkName "ID"$>$
]]$>$
$<$!ENTITY \% linkName "CDATA"$>$
$<$!ENTITY \% linkType "NAME"
        -- a list of these will be specified at a later date --$>$
$<$!ENTITY \% linkExtraAttributes
        "REL \%linkType \#IMPLIED
        REV \%linkType \#IMPLIED
        URN CDATA \#IMPLIED
        TITLE CDATA \#IMPLIED
        METHODS NAMES \#IMPLIED
        "$>$
$<$![ \%HTML.Recommended [
        $<$!ENTITY \% A.content   "(\%text)*"
        -- $<$H1$>$$<$a name="xxx"$>$Heading$<$/a$>$$<$/H1$>$
                is preferred to
           $<$a name="xxx"$>$$<$H1$>$Heading$<$/H1$>$$<$/a$>$
        --$>$
]]$>$
$<$!ENTITY \% A.content   "(\%heading|\%text)*"$>$
$<$!ELEMENT A     - - \%A.content -(A)$>$
$<$!ATTLIST A
        HREF \%URI \#IMPLIED
        NAME \%linkName \#IMPLIED
        \%linkExtraAttributes;
        \%SDAPREF; "$<$Anchor: \#AttList$>$"
        $>$
$<$!-- $<$A$>$               Anchor; source/destination of link    --$>$
$<$!-- $<$A NAME="..."$>$     Name of this anchor                --$>$
$<$!-- $<$A HREF="..."$>$     Address of link destination         --$>$
$<$!-- $<$A URN="..."$>$      Permanent address of destination     --$>$
$<$!-- $<$A REL=...$>$        Relationship to destination       --$>$
$<$!-- $<$A REV=...$>$        Relationship of destination to this    --$>$
$<$!-- $<$A TITLE="..."$>$    Title of destination (advisory)       --$>$
$<$!-- $<$A METHODS="..."$>$  Operations on destination (advisory)     --$>$
$<$!--========== Images ==========================--$>$
$<$!ELEMENT IMG    - O EMPTY$>$
$<$!ATTLIST IMG
        SRC \%URI;  \#REQUIRED
        ALT CDATA \#IMPLIED
        ALIGN (top|middle|bottom) \#IMPLIED
        ISMAP (ISMAP) \#IMPLIED
        \%SDAPREF; "$<$Fig$>$$<$?SDATrans Img: \#AttList$>$\#AttVal(Alt)$<$/Fig$>$"
        $>$
$<$!-- $<$IMG$>$              Image; icon, glyph or illustration      --$>$
$<$!-- $<$IMG SRC="..."$>$    Address of image object                 --$>$
$<$!-- $<$IMG ALT="..."$>$    Textual alternative                     --$>$
$<$!-- $<$IMG ALIGN=...$>$    Position relative to text               --$>$
$<$!-- $<$IMG ISMAP$>$        Each pixel can be a link                --$>$
$<$!--========== Paragraphs=======================--$>$
$<$!ELEMENT P     - O (\%text)*$>$
$<$!ATTLIST P
        \%SDAFORM; "Para"
        $>$
$<$!-- $<$P$>$        Paragraph       --$>$
$<$!--========== Headings, Titles, Sections ===============--$>$
$<$!ELEMENT HR    - O EMPTY$>$
$<$!ATTLIST HR
        \%SDAPREF; "\&\#RE;\&\#RE;"
        $>$
$<$!-- $<$HR$>$       Horizontal rule --$>$
$<$!ELEMENT ( \%heading )  - -  (\%text;)*$>$
$<$!ATTLIST H1
        \%SDAFORM; "H1"
        $>$
$<$!ATTLIST H2
        \%SDAFORM; "H2"
        $>$
$<$!ATTLIST H3
        \%SDAFORM; "H3"
        $>$
$<$!ATTLIST H4
        \%SDAFORM; "H4"
        $>$
$<$!ATTLIST H5
        \%SDAFORM; "H5"
        $>$
$<$!ATTLIST H6
        \%SDAFORM; "H6"
        $>$
$<$!-- $<$H1$>$       Heading, level 1 --$>$
$<$!-- $<$H2$>$       Heading, level 2 --$>$
$<$!-- $<$H3$>$       Heading, level 3 --$>$
$<$!-- $<$H4$>$       Heading, level 4 --$>$
$<$!-- $<$H5$>$       Heading, level 5 --$>$
$<$!-- $<$H6$>$       Heading, level 6 --$>$
$<$!--========== Text Flows ======================--$>$
$<$![ \%HTML.Forms [
        $<$!ENTITY \% block.forms "BLOCKQUOTE | FORM | ISINDEX"$>$
]]$>$
$<$!ENTITY \% block.forms "BLOCKQUOTE"$>$
$<$![ \%HTML.Deprecated [
        $<$!ENTITY \% preformatted "PRE | XMP | LISTING"$>$
]]$>$
$<$!ENTITY \% preformatted "PRE"$>$
$<$!ENTITY \% block "P | \%list | DL
        | \%preformatted
        | \%block.forms"$>$
$<$!ENTITY \% flow "(\%text|\%block)*"$>$
$<$!ENTITY \% pre.content "\#PCDATA | A | HR | BR"$>$
$<$!ELEMENT PRE - - (\%pre.content)*$>$
$<$!ATTLIST PRE
        WIDTH NUMBER \#implied
        \%SDAFORM; "Lit"
        $>$
$<$!-- $<$PRE$>$              Preformatted text               --$>$
$<$!-- $<$PRE WIDTH=...$>$    Maximum characters per line     --$>$
$<$![ \%HTML.Deprecated [
$<$!ENTITY \% literal "CDATA"
        -- historical, non-conforming parsing mode where
           the only markup signal is the end tag
           in full
        --$>$
$<$!ELEMENT (XMP|LISTING) - -  \%literal$>$
$<$!ATTLIST XMP
        \%SDAFORM; "Lit"
        \%SDAPREF; "Example:\&\#RE;"
        $>$
$<$!ATTLIST LISTING
        \%SDAFORM; "Lit"
        \%SDAPREF; "Listing:\&\#RE;"
        $>$
$<$!-- $<$XMP$>$              Example section         --$>$
$<$!-- $<$LISTING$>$          Computer listing        --$>$
$<$!ELEMENT PLAINTEXT - O \%literal$>$
$<$!-- $<$PLAINTEXT$>$        Plain text passage      --$>$
$<$!ATTLIST PLAINTEXT
        \%SDAFORM; "Lit"
        $>$
]]$>$
$<$!--========== Lists ==================--$>$
$<$!ELEMENT DL    - -  (DT | DD)+$>$
$<$!ATTLIST DL
        COMPACT (COMPACT) \#IMPLIED
        \%SDAFORM; "List"
        \%SDAPREF; "Definition List:"
        $>$
$<$!ELEMENT DT    - O (\%text)*$>$
$<$!ATTLIST DT
        \%SDAFORM; "Term"
        $>$
$<$!ELEMENT DD    - O \%flow$>$
$<$!ATTLIST DD
        \%SDAFORM; "LItem"
        $>$
$<$!-- $<$DL$>$               Definition list, or glossary    --$>$
$<$!-- $<$DL COMPACT$>$       Compact style list              --$>$
$<$!-- $<$DT$>$               Term in definition list         --$>$
$<$!-- $<$DD$>$               Definition of term              --$>$
$<$!ELEMENT (OL|UL) - -  (LI)+$>$
$<$!ATTLIST OL
        COMPACT (COMPACT) \#IMPLIED
        \%SDAFORM; "List"
        $>$
$<$!ATTLIST UL
        COMPACT (COMPACT) \#IMPLIED
        \%SDAFORM; "List"
        $>$
$<$!-- $<$UL$>$               Unordered list                  --$>$
$<$!-- $<$UL COMPACT$>$       Compact list style              --$>$
$<$!-- $<$OL$>$               Ordered, or numbered list       --$>$
$<$!-- $<$OL COMPACT$>$       Compact list style              --$>$
$<$!ELEMENT (DIR|MENU) - -  (LI)+ -(\%block)$>$
$<$!ATTLIST DIR
        COMPACT (COMPACT) \#IMPLIED
        \%SDAFORM; "List"
        \%SDAPREF; "$<$LHead$>$Directory$<$/LHead$>$"
        $>$
$<$!ATTLIST MENU
        COMPACT (COMPACT) \#IMPLIED
        \%SDAFORM; "List"
        \%SDAPREF; "$<$LHead$>$Menu$<$/LHead$>$"
        $>$
$<$!-- $<$DIR$>$              Directory list                  --$>$
$<$!-- $<$DIR COMPACT$>$      Compact list style              --$>$
$<$!-- $<$MENU$>$             Menu list                       --$>$
$<$!-- $<$MENU COMPACT$>$     Compact list style              --$>$
$<$!ELEMENT LI    - O \%flow$>$
$<$!ATTLIST LI
        \%SDAFORM; "LItem"
        $>$
$<$!-- $<$LI$>$               List item                       --$>$
$<$!--========== Document Body ===================--$>$
$<$![ \%HTML.Recommended [        
$<$!ENTITY \% body.content "(\%heading|\%block|HR|ADDRESS|IMG)*"        
-- $<$h1$>$Heading$<$/h1$>$        
   $<$p$>$Text ...               
is preferred to        
   $<$h1$>$Heading$<$/h1$>$        
   Text ...        
--$>$
]]$>$
$<$!ENTITY \% body.content "(\%heading | \%text | \%block |                          
 HR | ADDRESS)*"$>$
$<$!ELEMENT BODY O O  \%body.content$>$
$<$!-- $<$BODY$>$     Document body   --$>$
$<$!ELEMENT BLOCKQUOTE - - \%body.content$>$
$<$!ATTLIST BLOCKQUOTE
        \%SDAFORM; "BQ"
        $>$
$<$!-- $<$BLOCKQUOTE$>$       Quoted passage  --$>$
$<$!ELEMENT ADDRESS - - (\%text|P)*$>$
$<$!ATTLIST  ADDRESS
        \%SDAFORM; "Lit"
        \%SDAPREF; "Address:\&\#RE;"
        $>$
$<$!-- $<$ADDRESS$>$  Address, signature, or byline    --$>$
$<$!--======= Forms ====================--$>$
$<$![ \%HTML.Forms [
$<$!ELEMENT FORM - - \%body.content -(FORM) +(INPUT|SELECT|TEXTAREA)$>$
$<$!ATTLIST FORM
        ACTION \%URI \#IMPLIED
        METHOD (\%HTTP-Method) GET
        ENCTYPE \%Content-Type; "application/x-www-form-urlencoded"
        \%SDAPREF; "$<$Para$>$Form:$<$/Para$>$"
        \%SDASUFF; "$<$Para$>$Form End.$<$/Para$>$"
        $>$
$<$!-- $<$FORM$>$                     Fill-out or data-entry form     --$>$
$<$!-- $<$FORM ACTION="..."$>$        Address for completed form      --$>$
$<$!-- $<$FORM METHOD=...$>$          Method of submitting form       --$>$
$<$!-- $<$FORM ENCTYPE="..."$>$       Representation of form data     --$>$
$<$!ENTITY \% InputType "(TEXT | PASSWORD | CHECKBOX |
                        RADIO | SUBMIT | RESET |
                        IMAGE | HIDDEN )"$>$
$<$!ELEMENT INPUT - O EMPTY$>$
$<$!ATTLIST INPUT        
TYPE \%InputType TEXT        
NAME CDATA \#IMPLIED        
VALUE CDATA \#IMPLIED        
SRC \%URI \#IMPLIED        
CHECKED (CHECKED) \#IMPLIED        
SIZE CDATA \#IMPLIED        
MAXLENGTH NUMBER \#IMPLIED        
ALIGN (top|middle|bottom) \#IMPLIED
        \%SDAPREF; "Input: "        
$>$
$<$!-- $<$INPUT$>$         Form input datum         --$>$
$<$!-- $<$INPUT TYPE=...$>$     Type of input interaction        --$>$
$<$!-- $<$INPUT NAME=...$>$     Name of form datum             --$>$
$<$!-- $<$INPUT VALUE="..."$>$        Default/initial/selected value --$>$
$<$!-- $<$INPUT SRC="..."$>$   Address of image               --$>$
$<$!-- $<$INPUT CHECKED$>$       Initial state is "on"         --$>$
$<$!-- $<$INPUT SIZE=...$>$     Field size hint           --$>$
$<$!-- $<$INPUT MAXLENGTH=...$>$      Data length maximum   --$>$
$<$!-- $<$INPUT ALIGN=...$>$   Image alignment                --$>$
$<$!ELEMENT SELECT - - (OPTION+) -(INPUT|SELECT|TEXTAREA)$>$
$<$!ATTLIST SELECT
        NAME CDATA \#REQUIRED
        SIZE NUMBER \#IMPLIED
        MULTIPLE (MULTIPLE) \#IMPLIED
        \%SDAFORM; "List"
        \%SDAPREF;
        "$<$LHead$>$Select \#AttVal(Multiple)$<$/LHead$>$"        
$>$
$<$!-- $<$SELECT$>$      Selection of option(s)   --$>$
$<$!-- $<$SELECT NAME=...$>$   Name of form datum           --$>$
$<$!-- $<$SELECT SIZE=...$>$   Options displayed at a time     --$>$
$<$!-- $<$SELECT MULTIPLE$>$   Multiple selections allowed     --$>$
$<$!ELEMENT OPTION - O (\#PCDATA)*$>$
$<$!ATTLIST OPTION
        SELECTED (SELECTED) \#IMPLIED
        VALUE CDATA \#IMPLIED
        \%SDAFORM; "LItem"
        \%SDAPREF;
        "Option: \#AttVal(Value) \#AttVal(Selected)"        
$>$
$<$!-- $<$OPTION$>$      A selection option           --$>$
$<$!-- $<$OPTION SELECTED$>$   Initial state      --$>$
$<$!-- $<$OPTION VALUE="..."$>$       Form datum value for this option--$>$
$<$!ELEMENT TEXTAREA - - (\#PCDATA)* -(INPUT|SELECT|TEXTAREA)$>$
$<$!ATTLIST TEXTAREA
        NAME CDATA \#REQUIRED
        ROWS NUMBER \#REQUIRED
        COLS NUMBER \#REQUIRED
        \%SDAFORM; "Para"
        \%SDAPREF; "Input Text -- \#AttVal(Name): "
        $>$
$<$!-- $<$TEXTAREA$>$                An area for text input               --$>$
$<$!-- $<$TEXTAREA NAME=...$>$        Name of form datum         --$>$
$<$!-- $<$TEXTAREA ROWS=...$>$        Height of area                --$>$
$<$!-- $<$TEXTAREA COLS=...$>$        Width of area           --$>$
]]$>$
$<$!--======= Document Head ======================--$>$
$<$![ \%HTML.Recommended [        
$<$!ENTITY \% head.extra "META* \& LINK*"$>$
]]$>$
$<$!ENTITY \% head.extra "NEXTID? \& META* \& LINK*"$>$
$<$!ENTITY \% head.content "TITLE \& ISINDEX? \& BASE? \&                     
 (\%head.extra)"$>$
$<$!ELEMENT HEAD O O  (\%head.content)$>$
$<$!-- $<$HEAD$>$     Document head   --$>$
$<$!ELEMENT TITLE - -  (\#PCDATA)*$>$
$<$!ATTLIST TITLE
        \%SDAFORM; "Ti"    $>$
$<$!-- $<$TITLE$>$    Title of document --$>$
$<$!ELEMENT LINK - O EMPTY$>$
$<$!ATTLIST LINK
        HREF \%URI \#REQUIRED
        \%linkExtraAttributes;
        \%SDAPREF; "Linked to : \#AttVal (TITLE) (URN) (HREF)$>$"    $>$
$<$!-- $<$LINK$>$         Link from this document         --$>$
$<$!-- $<$LINK HREF="..."$>$  Address of link destination           --$>$
$<$!-- $<$LINK URN="..."$>$   Lasting name of destination             --$>$
$<$!-- $<$LINK REL=...$>$     Relationship to destination         --$>$
$<$!-- $<$LINK REV=...$>$     Relationship of destination to this         --$>$
$<$!-- $<$LINK TITLE="..."$>$ Title of destination (advisory)         --$>$
$<$!-- $<$LINK METHODS="..."$>$ Operations allowed (advisory)         --$>$
$<$!ELEMENT ISINDEX - O EMPTY$>$
$<$!ATTLIST ISINDEX
        \%SDAPREF;
   "$<$Para$>$[Document is indexed/searchable.]$<$/Para$>$"$>$
$<$!-- $<$ISINDEX$>$          Document is a searchable index          --$>$
$<$!ELEMENT BASE - O EMPTY$>$
$<$!ATTLIST BASE
        HREF \%URI; \#REQUIRED     $>$
$<$!-- $<$BASE$>$             Base context document                   --$>$
$<$!-- $<$BASE HREF="..."$>$  Address for this document               --$>$
$<$!ELEMENT NEXTID - O EMPTY$>$
$<$!ATTLIST NEXTID
        N \%linkName \#REQUIRED     $>$
$<$!-- $<$NEXTID$>$     Next ID to use for link name         --$>$
$<$!-- $<$NEXTID N=...$>$     Next ID to use for link name               --$>$
$<$!ELEMENT META - O EMPTY$>$
$<$!ATTLIST META
        HTTP-EQUIV  NAME    \#IMPLIED
        NAME        NAME    \#IMPLIED
        CONTENT     CDATA   \#REQUIRED    $>$
$<$!-- $<$META$>$                     Generic Metainformation         --$>$
$<$!-- $<$META HTTP-EQUIV=...$>$      HTTP response header name       --$>$
$<$!-- $<$META NAME=...$>$            Metainformation name            --$>$
$<$!-- $<$META CONTENT="..."$>$       Associated information          --$>$
$<$!--======= Document Structure =================--$>$
$<$![ \%HTML.Deprecated [
        $<$!ENTITY \% html.content "HEAD, BODY, PLAINTEXT?"$>$
]]$>$
$<$!ENTITY \% html.content "HEAD, BODY"$>$
$<$!ELEMENT HTML O O  (\%html.content)$>$
$<$!ENTITY \% version.attr "VERSION CDATA \#FIXED '\%HTML.Version;'"$>$
$<$!ATTLIST HTML
        \%version.attr;
        \%SDAFORM; "Book"
        $>$
$<$!-- $<$HTML$>$            HTML Document     --$>$
\subsection*{Strict HTML DTD}\par 
This document type declaration refers to the HTML DTD with the
{\tt `HTML.Recommended'} entity defined as {\tt `INCLUDE'} rather than
IGNORE;  that is, it refers to the more structurally rigid definition
of HTML.
\par $<$!--    html-s.dtd
        Document Type Definition for the HyperText Markup Language        
with strict validation (HTML Strict DTD).        
\$Id$        
Author: Daniel W. Connolly $<$connolly@w3.org$>$        
See Also: http://www.w3.org/hypertext/WWW/MarkUp/MarkUp.html
--$>$
$<$!ENTITY \% HTML.Version        
"-//IETF//DTD HTML 2.0 Strict//EN"
        -- Typical usage:
            $<$!DOCTYPE HTML PUBLIC               
"-//IETF//DTD HTML Strict//EN"$>$        
    $<$html$>$        
    ...        
    $<$/html$>$        
--        
$>$
$<$!-- Feature Test Entities --$>$
$<$!ENTITY \% HTML.Recommended "INCLUDE"$>$
$<$!ENTITY \% html PUBLIC "-//IETF//DTD HTML 2.0//EN"$>$
\%html;
\subsection*{Level 1 HTML DTD}\par 
This document type declaration refers to the HTML DTD with the
{\tt `HTML.Forms'} entity defined as {\tt `IGNORE'} rather than
{\tt `INCLUDE'}.  Documents which contain {\it FORM} elements do not
conform to this DTD, and must use the level 2 DTD.
\par $<$!--    html-1.dtd
        Document Type Definition for the HyperText Markup Language        
with Level 1 Extensions        (HTML Level 1 DTD).        
\$Id$        
Author: Daniel W. Connolly $<$connolly@w3.org$>$        
See Also: http://info.cern.ch/hypertext/WWW/MarkUp/MarkUp.html
--$>$
$<$!ENTITY \% HTML.Version        
"-//IETF//DTD HTML 2.0 Level 1//EN"
        -- Typical usage:
            $<$!DOCTYPE HTML PUBLIC               
"-//IETF//DTD HTML Level 1//EN"$>$        
    $<$html$>$        
    ...        
    $<$/html$>$        
--        
$>$
$<$!-- Feature Test Entities --$>$
$<$!ENTITY \% HTML.Forms "IGNORE"$>$
$<$!ENTITY \% html PUBLIC "-//IETF//DTD HTML 2.0//EN"$>$
\%html;
\subsection*{Strict Level 1 HTML DTD}\par 
This document type declaration refers to the level 1 HTML DTD with the
{\tt `HTML.Recommended'} entity defined as {\tt `INCLUDE'} rather than
IGNORE;  that is, it refers to the more structurally rigid definition
of HTML.
\par $<$!--    html-1s.dtd
        Document Type Definition for the HyperText Markup Language        
Struct Level 1        
\$Id$        
Author: Daniel W. Connolly $<$connolly@w3.org$>$        
See Also: http://www.w3.org/hypertext/WWW/MarkUp/MarkUp.html
--$>$
$<$!ENTITY \% HTML.Version        
"-//IETF//DTD HTML 2.0 Strict Level 1//EN"
        -- Typical usage:
            $<$!DOCTYPE HTML PUBLIC               
"-//IETF//DTD HTML Strict Level 1//EN"$>$        
    $<$html$>$        
    ...        
    $<$/html$>$        
--        
$>$
$<$!-- Feature Test Entities --$>$
$<$!ENTITY \% HTML.Recommended "INCLUDE"$>$
$<$!ENTITY \% html-1 PUBLIC "-//IETF//DTD HTML 2.0 Level 1//EN"$>$
\%html-1;
\subsection*{SGML Declaration for HTML}\par 
This is the SGML Declaration for HyperText Markup Language.
\par $<$!SGML  "ISO 8879:1986"
--        
SGML Declaration for HyperText Markup Language (HTML).
--
CHARSET
         BASESET  "ISO 646:1983//CHARSET
                   International Reference Version
                   (IRV)//ESC 2/5 4/0"
         DESCSET  0   9   UNUSED
                  9   2   9
                  11  2   UNUSED
                  13  1   13
                  14  18  UNUSED
                  32  95  32
                  127 1   UNUSED
     BASESET   "ISO Registration Number 100//CHARSET
                ECMA-94 Right Part of
                Latin Alphabet Nr. 1//ESC 2/13 4/1"
         DESCSET  128  32   UNUSED
                  160  96    32
CAPACITY        SGMLREF
                TOTALCAP        150000
                GRPCAP          150000               
ENTCAP               150000  
SCOPE    DOCUMENT
SYNTAX   
         SHUNCHAR CONTROLS 0 1 2 3 4 5 6 7 8 9 10 11 12 13 14 15 16               
 17 18 19 20 21 22 23 24 25 26 27 28 29 30 31 127
         BASESET  "ISO 646:1983//CHARSET
                   International Reference Version
                   (IRV)//ESC 2/5 4/0"
         DESCSET  0 128 0
         FUNCTION               
  RE          13
                  RS          10
                  SPACE       32
                  TAB SEPCHAR  9        
         NAMING   LCNMSTRT ""
                  UCNMSTRT ""
                  LCNMCHAR ".-"
                  UCNMCHAR ".-"
                  NAMECASE GENERAL YES
                           ENTITY  NO
         DELIM    GENERAL  SGMLREF
                  SHORTREF SGMLREF
         NAMES    SGMLREF
         QUANTITY SGMLREF
                  ATTSPLEN 2100
                  LITLEN   1024
                  NAMELEN  72    -- somewhat arbitrary; taken from
                                internet line length conventions --
                  PILEN    1024
                  TAGLEN   2100
                  GRPGTCNT 150
                  GRPCNT   64                   
FEATURES
  MINIMIZE
    DATATAG  NO
    OMITTAG  YES
    RANK     NO
    SHORTTAG YES
  LINK
    SIMPLE   NO
    IMPLICIT NO
    EXPLICIT NO
  OTHER
    CONCUR   NO
    SUBDOC   NO
    FORMAL   YES
  APPINFO    "SDA"  -- conforming SGML Document Access application               
    --
$>$
$<$!--         
\$Id$        
Author: Daniel W. Connolly $<$connolly@w3.org$>$        
See also: http://www.w3.org/hypertext/WWW/MarkUp/MarkUp.html
 --$>$
\subsection*{Sample SGML Open Entity Catalog for HTML}\par 
The SGML standard describes an "entity manager" as the portion or
component of an SGML system that maps SGML entities into the actual
storage model (e.g., the file system).  The standard itself does not
define a particular mapping methodology or notation.
\par \par 
To assist the interoperability among various SGML tools and
systems, the SGML Open consortium has passed a technical resolution
that defines a format for an application- independent entity catalog
that maps external identifiers and/or entity names to file names.
\par \par 
Each entry in the catalog associates a storage object identifier
(such as a file name) with information about the external entity that
appears in the SGML document.  In addition to entries that associate
public identifiers, a catalog entry can associate an entity name with
a storage object indentifier.  For example, the following are possible
catalog entries:
\par -- catalog: SGML Open style entity catalog for HTML --        
-- \$Id$ --        
-- Ways to refer to Level 2: most general to most specific --
PUBLIC  "-//IETF//DTD HTML//EN"                html.dtd
PUBLIC  "-//IETF//DTD HTML 2.0//EN"         html.dtd
PUBLIC  "-//IETF//DTD HTML Level 2//EN"         html.dtd
PUBLIC  "-//IETF//DTD HTML 2.0 Level 2//EN"     html.dtd        
-- Ways to refer to Level 1: most general to most specific --
PUBLIC  "-//IETF//DTD HTML Level 1//EN"         html-1.dtd
PUBLIC  "-//IETF//DTD HTML 2.0 Level 1//EN"     html-1.dtd        
-- Ways to refer to Level 0: most general to most specific --
PUBLIC  "-//IETF//DTD HTML Level 0//EN"         html-0.dtd
PUBLIC  "-//IETF//DTD HTML 2.0 Level 0//EN"     html-0.dtd        
-- Ways to refer to Strict Level 2: most general to most specific --
PUBLIC  "-//IETF//DTD HTML Strict//EN"           html-s.dtd
PUBLIC  "-//IETF//DTD HTML 2.0 Strict//EN"           html-s.dtd
PUBLIC  "-//IETF//DTD HTML Strict Level 2//EN"   html-s.dtd
PUBLIC  "-//IETF//DTD HTML 2.0 Strict Level 2//EN"      html-s.dtd        
-- Ways to refer to Strict Level 1: most general to most specific --
PUBLIC  "-//IETF//DTD HTML Strict Level 1//EN"   html-1s.dtd
PUBLIC  "-//IETF//DTD HTML 2.0 Strict Level 1//EN"      html-1s.dtd        
-- Ways to refer to Strict Level 0: most general to most specific --
PUBLIC  "-//IETF//DTD HTML Strict Level 0//EN"   html-0s.dtd
PUBLIC  "-//IETF//DTD HTML 2.0 Strict Level 0//EN"      html-0s.dtd        
-- ISO latin 1 entity set for HTML -- 
PUBLIC  "ISO 8879-1986//ENTITIES Added Latin 1//EN//HTML"        ISOlat1.sgml
\subsection*{Character Entity Sets}\par 
The HTML DTD defines the following entities.  They represent
particular graphic characters which have special meanings in places in
the markup, or may not be part of the character set available to the
writer.
\par \subsubsection*{Numeric and Special Graphic Entity Set}\par 
The following table lists each of the characters included from the
Numeric and Special Graphic entity set, along with its name, syntax
for use, and description.  This list is derived from {\tt `ISO Standard 8879:1986//ENTITIES Numeric and Special Graphic//EN'}.  However,
HTML does not include for the entire entity set -- only the entities
listed below are included.
\par GLYPH   NAME      SYNTAX       DESCRIPTION
$<$       lt      \&lt;    Less than sign
$>$       gt      \&gt;    Greater than sign
\&       amp     \&amp;   Ampersand
"       quot    \&quot;  Double quote sign
\subsubsection*{ISO Latin 1 Character Entity Set}\par 
The following public text lists each of the characters specified in the
Added Latin 1 entity set, along with its name, syntax for use, and
description.  This list is derived from ISO Standard
8879:1986//ENTITIES Added Latin 1//EN.  HTML includes the entire entity
set.
\par $<$!-- (C) International Organization for Standardization 1986
     Permission to copy in any form is granted for use with
     conforming SGML systems and applications as defined in
     ISO 8879, provided this notice is included in all copies.
--$>$
$<$!-- Character entity set. Typical invocation:
     $<$!ENTITY \% ISOlat1 PUBLIC
       "ISO 8879-1986//ENTITIES Added Latin 1//EN//HTML"$>$
     \%ISOlat1;
--$>$
$<$!--    Modified for use in HTML        
\$Id$ --$>$
$<$!ENTITY AElig  CDATA "\&\#198;" -- capital AE diphthong (ligature) --$>$
$<$!ENTITY Aacute CDATA "\&\#193;" -- capital A, acute accent --$>$
$<$!ENTITY Acirc  CDATA "\&\#194;" -- capital A, circumflex accent --$>$
$<$!ENTITY Agrave CDATA "\&\#192;" -- capital A, grave accent --$>$
$<$!ENTITY Aring  CDATA "\&\#197;" -- capital A, ring --$>$
$<$!ENTITY Atilde CDATA "\&\#195;" -- capital A, tilde --$>$
$<$!ENTITY Auml   CDATA "\&\#196;" -- capital A, dieresis or umlaut mark --$>$
$<$!ENTITY Ccedil CDATA "\&\#199;" -- capital C, cedilla --$>$
$<$!ENTITY ETH    CDATA "\&\#208;" -- capital Eth, Icelandic --$>$
$<$!ENTITY Eacute CDATA "\&\#201;" -- capital E, acute accent --$>$
$<$!ENTITY Ecirc  CDATA "\&\#202;" -- capital E, circumflex accent --$>$
$<$!ENTITY Egrave CDATA "\&\#200;" -- capital E, grave accent --$>$
$<$!ENTITY Euml   CDATA "\&\#203;" -- capital E, dieresis or umlaut mark --$>$
$<$!ENTITY Iacute CDATA "\&\#205;" -- capital I, acute accent --$>$
$<$!ENTITY Icirc  CDATA "\&\#206;" -- capital I, circumflex accent --$>$
$<$!ENTITY Igrave CDATA "\&\#204;" -- capital I, grave accent --$>$
$<$!ENTITY Iuml   CDATA "\&\#207;" -- capital I, dieresis or umlaut mark --$>$
$<$!ENTITY Ntilde CDATA "\&\#209;" -- capital N, tilde --$>$
$<$!ENTITY Oacute CDATA "\&\#211;" -- capital O, acute accent --$>$
$<$!ENTITY Ocirc  CDATA "\&\#212;" -- capital O, circumflex accent --$>$
$<$!ENTITY Ograve CDATA "\&\#210;" -- capital O, grave accent --$>$
$<$!ENTITY Oslash CDATA "\&\#216;" -- capital O, slash --$>$
$<$!ENTITY Otilde CDATA "\&\#213;" -- capital O, tilde --$>$
$<$!ENTITY Ouml   CDATA "\&\#214;" -- capital O, dieresis or umlaut mark --$>$
$<$!ENTITY THORN  CDATA "\&\#222;" -- capital THORN, Icelandic --$>$
$<$!ENTITY Uacute CDATA "\&\#218;" -- capital U, acute accent --$>$
$<$!ENTITY Ucirc  CDATA "\&\#219;" -- capital U, circumflex accent --$>$
$<$!ENTITY Ugrave CDATA "\&\#217;" -- capital U, grave accent --$>$
$<$!ENTITY Uuml   CDATA "\&\#220;" -- capital U, dieresis or umlaut mark --$>$
$<$!ENTITY Yacute CDATA "\&\#221;" -- capital Y, acute accent --$>$
$<$!ENTITY aacute CDATA "\&\#225;" -- small a, acute accent --$>$
$<$!ENTITY acirc  CDATA "\&\#226;" -- small a, circumflex accent --$>$
$<$!ENTITY aelig  CDATA "\&\#230;" -- small ae diphthong (ligature) --$>$
$<$!ENTITY agrave CDATA "\&\#224;" -- small a, grave accent --$>$
$<$!ENTITY aring  CDATA "\&\#229;" -- small a, ring --$>$
$<$!ENTITY atilde CDATA "\&\#227;" -- small a, tilde --$>$
$<$!ENTITY auml   CDATA "\&\#228;" -- small a, dieresis or umlaut mark --$>$
$<$!ENTITY ccedil CDATA "\&\#231;" -- small c, cedilla --$>$
$<$!ENTITY eacute CDATA "\&\#233;" -- small e, acute accent --$>$
$<$!ENTITY ecirc  CDATA "\&\#234;" -- small e, circumflex accent --$>$
$<$!ENTITY egrave CDATA "\&\#232;" -- small e, grave accent --$>$
$<$!ENTITY eth    CDATA "\&\#240;" -- small eth, Icelandic --$>$
$<$!ENTITY euml   CDATA "\&\#235;" -- small e, dieresis or umlaut mark --$>$
$<$!ENTITY iacute CDATA "\&\#237;" -- small i, acute accent --$>$
$<$!ENTITY icirc  CDATA "\&\#238;" -- small i, circumflex accent --$>$
$<$!ENTITY igrave CDATA "\&\#236;" -- small i, grave accent --$>$
$<$!ENTITY iuml   CDATA "\&\#239;" -- small i, dieresis or umlaut mark --$>$
$<$!ENTITY ntilde CDATA "\&\#241;" -- small n, tilde --$>$
$<$!ENTITY oacute CDATA "\&\#243;" -- small o, acute accent --$>$
$<$!ENTITY ocirc  CDATA "\&\#244;" -- small o, circumflex accent --$>$
$<$!ENTITY ograve CDATA "\&\#242;" -- small o, grave accent --$>$
$<$!ENTITY oslash CDATA "\&\#248;" -- small o, slash --$>$
$<$!ENTITY otilde CDATA "\&\#245;" -- small o, tilde --$>$
$<$!ENTITY ouml   CDATA "\&\#246;" -- small o, dieresis or umlaut mark --$>$
$<$!ENTITY szlig  CDATA "\&\#223;" -- small sharp s, German (sz ligature) --$>$
$<$!ENTITY thorn  CDATA "\&\#254;" -- small thorn, Icelandic --$>$
$<$!ENTITY uacute CDATA "\&\#250;" -- small u, acute accent --$>$
$<$!ENTITY ucirc  CDATA "\&\#251;" -- small u, circumflex accent --$>$
$<$!ENTITY ugrave CDATA "\&\#249;" -- small u, grave accent --$>$
$<$!ENTITY uuml   CDATA "\&\#252;" -- small u, dieresis or umlaut mark --$>$
$<$!ENTITY yacute CDATA "\&\#253;" -- small y, acute accent --$>$
$<$!ENTITY yuml   CDATA "\&\#255;" -- small y, dieresis or umlaut mark --$>$
\section*{Terms}\begin{description}\item[{\it absolute URI}]
a URI in absolute form, as per [URL]
\item[{\it anchor}]
a hyperlink navigation option;  typically, a highlighted
phrase marked as an {\it A} element.
\item[{\it base URI}]
URI used as the base of an HTML document for the
purpose of resolving hyperlink destinations.
\item[{\it character}]
An atom of information, for example a letter or a
digit.  Graphic characters have associated glyphs, where as control
characters have associated processing semantics.
\item[{\it character encoding scheme}]
A function whose domain is the set of
sequences of octets, and whose range is the set of sequences of
characters from a character repertoire;  that is, a sequence of octets
and a character encoding scheme determines a sequence of characters.
\item[{\it character repertoire}]
A finite set of characters;  e.g.  the range
of a coded character set.
\item[{\it code position}]
An integer.  A coded character set and a code
position from its domain determine a character.
\item[{\it coded character set}]
A function whose domain is a subset of the
integers and whose range is a character repertoire.  That is, for some
set of integers (usually of the form \{0, 1, 2, ..., N\} ), a coded
character set and an integer in that set determine a
character.  Conversely, a character and a coded character set determine
the character's code position (or, in rare cases, a few code
positions).
\item[{\it conforming HTML user agent}]
A user agent that conforms to this
specification in its processing of the Internet Media Type
{\tt `text/html'}.
\item[{\it data character}]
Characters other than markup, which make up the
content of elements.
\item[{\it document character set}]
a coded character set whose range
includes all characters used in a document.  Every SGML document has
exactly one document character set.  Numeric character references are
resolved via the document character set.
\item[{\it DTD}]
document type definition.  Rules that apply SGML to the
markup of documents of a particular type, including a set of element
and entity declarations.  [SGML]
\item[{\it element}]
A component of the hierarchical structure defined by a document type definition;  it is identified in a document instance by descriptive markup, sually a start-tag and end-tag.  [SGML]
\item[{\it end-tag}]
Descriptive markup that identifies the end of an
element.  [SGML]
\item[{\it entity}]
data with an associated notation or interpretation;  for
example, a sequence of octets associated with an Internet Media
Type.[SGML]
\item[{\it fragment identifier}]
the portion of an {\it HREF} attribute
value following the {\tt `\#'} character which modifies the prenentation
of the destination of a hyperlink.
\item[{\it form data set}]
a sequence of name/value pairs;  the names are
given by an HTML document and the values are given by a user.
\item[{\it HTML document}]
An SGML document conforming to this document type
definition.
\item[{\it hyperlink}]
a relationship between to resources, called the source
and the destination.
\item[{\it markup}]
Syntactically delimited characters added to the data of a
document to represent its structure.  There are four different kinds of
markup:  descriptive markup (tags), references, markup declarations,
and processing instructions.[SGML]
\item[{\it may}]
A document or user interface is conforming whether this
statement applies or not.
\item[{\it media type}]
an Internet Media Type, as per [IMEDIA].
\item[{\it message entity}]
a head and body.  The head is a collection of
name/value fields, and the body is a sequence of octets.  The head
defines the content type and content transfer encoding of the body.  [MIME]
\item[{\it minimally conforming HTML user agent}]
A user agent that conforms
to this specification except for form processing.  It may only process
level 1 HTML documents.
\item[{\it must}]
Documents or user agents in conflict with this statement
are not conforming.
\item[{\it SGML document}]
A sequence of characters organized physically as a
set of entities and logically into a hierarchy of elements.  An SGML
document consists of data characters and markup;  the markup describes
the structure of the information and an instance of that
structure.[SGML]
\item[{\it shall}]
If a document or user agent conflicts with this statement,
it does not conform to this specification.
\item[{\it should}]
If a document or user agent conflicts with this
statement, undesirable results may occur in practice even though it
conforms to this specification.
\item[{\it start-tag}]
Descriptive markup that identifies the start of an
element and specifies its generic identifier and attributes.  [SGML]
\item[{\it syntax-reference character set}]
A coded character set whose range
includes all characters used for markup;  e.g.  name characters and
delimiter characters.
\item[{\it tag}]
Markup that delimits an element.  A tag includes a name which
refers to an element declaration in the DTD, and may include
attributes.[SGML]
\item[{\it text entity}]
A finite sequence of characters.  A text entity
typically takes the form of a sequence of octets with some associated
character encoding scheme, transmitted over the network or stored in a
file.[SGML]
\item[{\it typical}]
Typical processing is described for many elements.  This
is not a mandatory part of the specification but is given as guidance
for designers and to help explain the uses for which the elements were
intended.
\item[{\it URI}]
A Universal Resource Identifier is a formatted string that
serves as an identifier for a resource, typcally on the Internet.  URIs
are used in HTML to identify the destination of hyperlinks.  URIs in
common practice include Uniform Resource Locators (URLs)[URL]
and Relative URLs[RELURL].
\item[{\it user agent}]
A component of a distributed system that presents
an interface and processes requests on behalf of a user;  for example,
a www browser or a mail user agent.
\item[{\it WWW}]
The World-Wide Web is a hypertext-based, distributed
information system created by researchers at CERN in
Switzerland.  Users may create, edit or browse hypertext
documents.  {\tt `http://www.w3.org/'}\end{description}\section*{References}\begin{description}\item[{\it [URI]}]
T.  Berners-Lee.  "Universal Resource Identifiers in WWW:  A
Unifying Syntax for the Expression of Names and Addresses of Objects
on the Network as used in the World- Wide Web." RFC 1630, CERN, June 1994.
\item[{\it [URL]}]
T.  Berners-Lee, L.  Masinter, and M.  McCahill.  "Uniform
Resource Locators (URL)." RFC 1738, CERN,
Xerox PARC, University of Minnesota, October 1994.
\item[{\it [HTTP]}]
T.  Berners-Lee, R.  T.  Fielding, and H.  Frystyk
Nielsen.  "Hypertext Transfer Protocol - HTTP/1.0." Work in Progress
(draft-ietf-http-v10-spec-00.ps), MIT,
UC Irvine, CERN, March 1995.
\item[{\it [MIME]}]
N.  Borenstein and N.  Freed.  "MIME (Multipurpose Internet
Mail Extensions) Part One:  Mechanisms for Specifying and Describing
the Format of Internet Message Bodies." RFC
1521, Bellcore, Innosoft, September 1993.
\item[{\it [RELURL]}]
R.  T.  Fielding.  "Relative Uniform Resource Locators."
Work in Progress (draft-ietf-uri-relative-url-06.txt), UC Irvine, March 1995.
\item[{\it [GOLD90]}]
C.  F.  Goldfarb.  "The SGML Handbook." Y.  Rubinsky, Ed.,
Oxford University Press, 1990.
\item[{\it [DEXTER]}]
Frank Halasz and Mayer Schwartz, "The Dexter Hypertext
Reference Model", "Communications of the ACM", pp.  30-39, vol.  37
no.  2, Feb 1994,
\item[{\it [IMEDIA]}]
J.  Postel.  "Media Type Registration Procedure."
RFC 1590, USC/ISI, March 1994.
\item[{\it [IANA]}]
J.  Reynolds and J.  Postel.  "Assigned Numbers." STD 2,
RFC 1700, USC/ISI, October 1994.
\item[{\it [SQ91]}]
SoftQuad.  "The SGML Primer." 3rd ed., SoftQuad Inc., 1991.
\item[{\it [US-ASCII]}]
US-ASCII.  Coded Character Set - 7-Bit American
Standard Code for Information Interchange.  Standard ANSI X3.4-1986,
ANSI, 1986.
\item[{\it [ISO-8859-1]}]
ISO
8859.  International Standard -- Information Processing -- 8-bit
Single-Byte Coded Graphic Character Sets -- Part 1:  Latin Alphabet
No.  1, ISO 8859-1:1987.  Part 2:  Latin alphabet No.  2, ISO 8859-2,
1987.  Part 3:  Latin alphabet No.  3, ISO 8859-3, 1988.  Part 4:  Latin
alphabet No.  4, ISO 8859-4, 1988.  Part 5:  Latin/Cyrillic alphabet, ISO
8859-5, 1988.  Part 6:  Latin/Arabic alphabet, ISO 8859-6, 1987.  Part 7:
Latin/Greek alphabet, ISO 8859-7, 1987.  Part 8:  Latin/Hebrew alphabet,
ISO 8859-8, 1988.  Part 9:  Latin alphabet No.  5, ISO 8859-9, 1990.
\item[{\it [SGML]}]
ISO 8879.  Information
Processing - Text and Office Systems - Standard Generalized Markup
Language (SGML), 1986.
\end{description}\section*{Acknowledgments}\par 
The HTML document type was designed by Tim Berners-Lee at CERN as
part of the 1990 World Wide Web project.  In 1992, Dan Connolly wrote
the HTML Document Type Definition (DTD) and a brief HTML
specification.
\par \par 
Since 1993, a wide variety of Internet participants have
contributed to the evolution of HTML, which has included the addition
of in-line images introduced by the NCSA Mosaic software for WWW.  Dave
Raggett played an important role in deriving the FORMS material from
the HTML+ specification.
\par \par 
Dan Connolly and Karen Olson Muldrow rewrote the HTML Specification
in 1994.  The document was then edited by the HTML working group as a
whole, with updates being made by Eric Schieler, Mike Knezovich, and
Eric W.  Sink at Spyglass, Inc.  Finally, Roy Fielding restructured the
entire draft into its current form.
\par \par 
Special thanks to the many active participants in the HTML working
group, too numerous to list individually, without whom there would be
no standards process and no standard.  That this document approaches
its objective of carefully converging a description of current
practice and formalization of HTML's relationship to SGML is a tribute
to their effort.
\par \subsection*{Authors' Addresses}\begin{description}\item[{\it Tim Berners-Lee}]Director, W3 Consortium
MIT Laboratory for Computer Science
545 Technology Square
Cambridge, MA 02139, U.S.A.
Tel: +1 (617) 253 9670
Fax: +1 (617) 258 8682
Email: timbl@w3.org
\item[{\it Daniel W.  Connolly}]Research Technical Staff, W3 Consortium
MIT Laboratory for Computer Science
545 Technology Square
Cambridge, MA 02139, U.S.A.
Fax: +1 (617) 258 8682
Email: connolly@w3.org
URI: http://www.w3.org/hypertext/WWW/People/Connolly/
\end{description}\section*{The ANSI/ISO 8859-1 Coded Character Set}\par 
This list, sorted numerically, is derived from ANSI/ISO 8859-1 8-bit
single-byte coded graphic character set:
\par REFERENCE       DESCRIPTION
\&\#00; - \&\#08;   Unused
\&\#09;     Horizontal tab
\&\#10;     Line feed
\&\#11; - \&\#12;   Unused
\&\#13;     Carriage Return
\&\#14; - \&\#31;   Unused
\&\#32;     Space
\&\#33;     Exclamation mark
\&\#34;     Quotation mark
\&\#35;     Number sign
\&\#36;     Dollar sign
\&\#37;     Percent sign
\&\#38;     Ampersand
\&\#39;     Apostrophe
\&\#40;     Left parenthesis
\&\#41;     Right parenthesis
\&\#42;     Asterisk
\&\#43;     Plus sign
\&\#44;     Comma
\&\#45;     Hyphen
\&\#46;     Period (fullstop)
\&\#47;     Solidus (slash)
\&\#48; - \&\#57;   Digits 0-9
\&\#58;     Colon
\&\#59;     Semi-colon
\&\#60;     Less than
\&\#61;     Equals sign
\&\#62;     Greater than
\&\#63;     Question mark
\&\#64;     Commercial at
\&\#65; - \&\#90;   Letters A-Z
\&\#91;     Left square bracket
\&\#92;     Reverse solidus (backslash)
\&\#93;     Right square bracket
\&\#94;     Caret
\&\#95;     Horizontal bar (underscore)
\&\#96;     Acute accent
\&\#97; - \&\#122;  Letters a-z
\&\#123;   Left curly brace
\&\#124;   Vertical bar
\&\#125;   Right curly brace
\&\#126;   Tilde
\&\#127; - \&\#160; Unused
\&\#161;   Inverted exclamation
\&\#162;   Cent sign
\&\#163;   Pound sterling
\&\#164;   General currency sign
\&\#165;   Yen sign
\&\#166;   Broken vertical bar
\&\#167;   Section sign
\&\#168;   Umlaut (dieresis)
\&\#169;   Copyright
\&\#170;   Feminine ordinal
\&\#171;   Left angle quote, guillemotleft
\&\#172;   Not sign
\&\#173;   Soft hyphen
\&\#174;   Registered trademark
\&\#175;   Macron accent
\&\#176;   Degree sign
\&\#177;   Plus or minus
\&\#178;   Superscript two
\&\#179;   Superscript three
\&\#180;   Acute accent
\&\#181;   Micro sign
\&\#182;   Paragraph sign
\&\#183;   Middle dot
\&\#184;   Cedilla
\&\#185;   Superscript one
\&\#186;   Masculine ordinal
\&\#187;   Right angle quote, guillemotright
\&\#188;   Fraction one-fourth
\&\#189;   Fraction one-half
\&\#190;   Fraction three-fourths
\&\#191;   Inverted question mark
\&\#192;   Capital A, grave accent
\&\#193;   Capital A, acute accent
\&\#194;   Capital A, circumflex accent
\&\#195;   Capital A, tilde
\&\#196;   Capital A, dieresis or umlaut mark
\&\#197;   Capital A, ring
\&\#198;   Capital AE dipthong (ligature)
\&\#199;   Capital C, cedilla
\&\#200;   Capital E, grave accent
\&\#201;   Capital E, acute accent
\&\#202;   Capital E, circumflex accent
\&\#203;   Capital E, dieresis or umlaut mark
\&\#204;   Capital I, grave accent
\&\#205;   Capital I, acute accent
\&\#206;   Capital I, circumflex accent
\&\#207;   Capital I, dieresis or umlaut mark
\&\#208;   Capital Eth, Icelandic
\&\#209;   Capital N, tilde
\&\#210;   Capital O, grave accent
\&\#211;   Capital O, acute accent
\&\#212;   Capital O, circumflex accent
\&\#213;   Capital O, tilde
\&\#214;   Capital O, dieresis or umlaut mark
\&\#215;   Multiply sign
\&\#216;   Capital O, slash
\&\#217;   Capital U, grave accent
\&\#218;   Capital U, acute accent
\&\#219;   Capital U, circumflex accent
\&\#220;   Capital U, dieresis or umlaut mark
\&\#221;   Capital Y, acute accent
\&\#222;   Capital THORN, Icelandic
\&\#223;   Small sharp s, German (sz ligature)
\&\#224;   Small a, grave accent
\&\#225;   Small a, acute accent
\&\#226;   Small a, circumflex accent
\&\#227;   Small a, tilde
\&\#228;   Small a, dieresis or umlaut mark
\&\#229;   Small a, ring
\&\#230;   Small ae dipthong (ligature)
\&\#231;   Small c, cedilla
\&\#232;   Small e, grave accent
\&\#233;   Small e, acute accent
\&\#234;   Small e, circumflex accent
\&\#235;   Small e, dieresis or umlaut mark
\&\#236;   Small i, grave accent
\&\#237;   Small i, acute accent
\&\#238;   Small i, circumflex accent
\&\#239;   Small i, dieresis or umlaut mark
\&\#240;   Small eth, Icelandic
\&\#241;   Small n, tilde
\&\#242;   Small o, grave accent
\&\#243;   Small o, acute accent
\&\#244;   Small o, circumflex accent
\&\#245;   Small o, tilde
\&\#246;   Small o, dieresis or umlaut mark
\&\#247;   Division sign
\&\#248;   Small o, slash
\&\#249;   Small u, grave accent
\&\#250;   Small u, acute accent
\&\#251;   Small u, circumflex accent
\&\#252;   Small u, dieresis or umlaut mark
\&\#253;   Small y, acute accent
\&\#254;   Small thorn, Icelandic
\&\#255;   Small y, dieresis or umlaut mark
\section*{Proposed Entities}\par 
The HTML DTD references the "Added Latin 1" entity set, which only
supplies named entities for a subset of the non-ASCII characters in
ISO 8859-1, namely the accented characters.  The following entities
should be supported so that the remaining characters may only be
referenced symbolically.
\par $<$!ENTITY yuml   CDATA "\&\#255;" -- small y, dieresis or umlaut mark --$>$   
$<$!ENTITY iexcl   CDATA "\&\#161;" -- inverted exclamation mark  --$>$
$<$!ENTITY cent    CDATA "\&\#162" -- cent sign  --$>$
$<$!ENTITY pound   CDATA "\&\#163" -- pound sterling sign  --$>$
$<$!ENTITY curren  CDATA "\&\#164" -- general currency sign  --$>$
$<$!ENTITY yen     CDATA "\&\#165" -- yen sign  --$>$
$<$!ENTITY brvbar  CDATA "\&\#166" -- broken (vertical) bar  --$>$
$<$!ENTITY sect    CDATA "\&\#167" -- section sign  --$>$
$<$!ENTITY umlaut  CDATA "\&\#168" -- umlaut (dieresis)  --$>$
$<$!ENTITY copy    CDATA "\&\#169" -- copyright sign  --$>$
$<$!ENTITY ordf    CDATA "\&\#170" -- ordinal indicator, feminine  --$>$
$<$!ENTITY laquo   CDATA "\&\#171" -- angle quotation mark, left  --$>$
$<$!ENTITY not     CDATA "\&\#172" -- not sign  --$>$
$<$!ENTITY shy     CDATA "\&\#173" -- soft hyphen  --$>$
$<$!ENTITY reg     CDATA "\&\#174" -- registered trademark  --$>$
$<$!ENTITY macron  CDATA "\&\#175" -- macron  --$>$
$<$!ENTITY deg     CDATA "\&\#176" -- degree sign  --$>$
$<$!ENTITY plusmn  CDATA "\&\#177" -- plus-or-minus sign  --$>$
$<$!ENTITY sup2    CDATA "\&\#178" -- superscript two  --$>$
$<$!ENTITY sup3    CDATA "\&\#179" -- superscript three  --$>$
$<$!ENTITY acute   CDATA "\&\#180" -- acute accent  --$>$
$<$!ENTITY micro   CDATA "\&\#181" -- micro sign  --$>$
$<$!ENTITY para    CDATA "\&\#182" -- pilcrow (paragraph sign)  --$>$
$<$!ENTITY middot  CDATA "\&\#183" -- middle dot (centred decimal point)  --$>$
$<$!ENTITY cedilla CDATA "\&\#184" -- cedilla accent  --$>$
$<$!ENTITY sup1    CDATA "\&\#185" -- superscript one --$>$
$<$!ENTITY ordm    CDATA "\&\#186" -- ordinal indicator, masculine --$>$
$<$!ENTITY raquo   CDATA "\&\#187" -- angle quotation mark, right --$>$
$<$!ENTITY frac14  CDATA "\&\#188" -- fraction one-quarter --$>$
$<$!ENTITY frac12  CDATA "\&\#189" -- fraction one-half --$>$
$<$!ENTITY frac34  CDATA "\&\#190" -- fraction three-quarters --$>$
$<$!ENTITY iquest  CDATA "\&\#191" -- inverted question mark --$>$
$<$!ENTITY times   CDATA "\&\#215" -- multiply sign --$>$
$<$!ENTITY divide  CDATA "\&\#247" -- divide sign --$>$
\end{document}
